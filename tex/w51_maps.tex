/Users/adam/work/h2co/maps/paper/preface.tex

\newcommand{\casa}{2}
\newcommand{\eso}{1}
\newcommand{\cfa}{3}
\newcommand{\yale}{4}

\author{Adam Ginsburg\altaffilmark{\eso},
        John Bally\altaffilmark{\casa},
        Cara Battersby\altaffilmark{\cfa},
        Jeremy Darling\altaffilmark{\casa},
        Allison Youngblood\altaffilmark{\casa},
        Mayra Santos\altaffilmark{\yale},
        Hector Arce\altaffilmark{\yale},
        and probably a few others (Roberto Galvan-Madrid, Jonathan Tan)
        }
\email{Adam.G.Ginsburg@gmail.com}

\affil{{$^\casa$}{\it{CASA, University of Colorado, 389-UCB, Boulder, CO 80309}}}
\affil{{$^\eso$}{\it{European Southern Observatory, Karl-Schwarzschild-Strasse 2, D-85748 Garching bei München, Germany}}}
\affil{{$^\monash$}{\it{Monash Centre for Astrophysics, School of Mathematical Sciences, Monash University, Vic 3800, Australia}}}
\affil{{$^\cfa$}{\it{CfA}}}
\affil{{$^\yale$}{\it{Yale}}}



% Somewhere in here:
% need a shoutout to "Overheard on Astro-PH"
%
% FOCUS ON THE *OBSERVATIONAL* QUESTIONS solved here... the theoretical just
% aren't going to be that exciting

\section{Abstract}
We present new 2 cm and 6 cm maps of \formaldehyde and the radio continuum in
the W51 star forming complex acquired with Arecibo and the Green Bank Telescope
at $\sim50\arcsec$ resolution.
We perform a detailed analysis of the data and data processing with an emphasis on
ensuring the reliability of the data.
The data set has been made public \todo{if someone is willing to host}


\section{Introduction}
\todo{To do.  To-do items are coded in red.}

Massive star clusters are among the most visually outstanding features in the
night sky.  In other galaxies, they are useful probes of the star formation history
and can be individually identified and measured \citep{Bastian2008a}.  Locally,
they are the essential laboratories in which we can study the formation of massive
stars \citep{Davies2012a}.

In order to utilize these clusters as laboratories, we need to understand their
formation in detail.  Clusters are often assumed and measured to be coeval
\citep[e.g.][]{Kudryavtseva2012a}, but uncertainties remain \citep{Beccari2010a}.
In the most massive clusters, there are predictions that multiple generations
or an extended generation of stars should form prior to gas expulsion because
the gas will remain gravitationally bound \citep{Bressert2012a}.   Feedback from
and within young massive clusters is an active field of numerical study
\citep{Rogers2013a,Dale2013a,Dale2012a,Dale2008a,Dale2005a,Parker2013a,Myers2014a,Krumholz2014a}
%While
%only 5-35\% of all stars form in bound clusters \citep{Kruijssen2012a},

The results of cluster formation may be decided before the first stars are
formed.  The starless initial conditions of massive clusters have not yet been
definitively observed \citep{Ginsburg2012a} though there are viable candidates
such as G0.253+0.016 \citep{Longmore2012b}.  The initial conditions for star
formation on any scale are clearly turbulent.  However, there is no evidence
whether these initial conditions differ in any qualitative way from turbulence
in local, low-mass star-forming regions.



\section{Observations \& Data Reduction}

(observations paragraph)
The W51 survey was performed in September 2011 and 2012.  The GBT data were
taken as part of program AGBT10B/019; the raw data are available from the NRAO
archive\url{https://archive.nrao.edu/}.  The data presented in this paper
include sessions 10, 11, 14, 16, 17, 20, 21, and 22; the other sessions from
this project include maps of outer galaxy regions and a single-pointing survey
of Galactic plane sources that will be presented in another paper. 

The Arecibo data were taken as part of project A2705 over the course of 4
nights, September 10, 11, 12, and 15 2012.  On the first night, September 10
2012, a significant fraction of the data was lost due to an internal instrument
error within the Mock spectrometer, which resulted in a loss of the
high-resolution component of the \formaldehyde data for that night.  As a
result, we have focused our study on the lower-resolution ($\sim1 \kms$) data.

Data were taken in on-the-fly mode with the GBT Ku-band dual-beam system and
the Arecibo C-band receiver.  On the GBT, cross-hatched north-south and
east-west maps were created in Galactic coordinates, while at Arecibo only
east-west maps were taken.

The data was reduced using custom-made scripts based off of both GBTIDL's
mapping routines (cite Langston?) and Phil Perillat's AOIDL routines.  The code
is available at \url{https://github.com/keflavich/sdpy}.  The data reduction
workflow, along with a summary of the code, is presented in the Appendix.

The individual spectral were calibrated using a noise calibration diode as
usual.  For the GBT data, the first and last scans of each observation were
used as off positions, and the background level to be subtracted off of the
continuum was determined by linearly interpolating between these scans.

The GBT data infrequently exhibited major data artifacts; a key phase of the
reduction process was excising bad observations, which were usually isolated to
a single component of the backend.

The Arecibo data reduction process for W51
presented unique challenges: at C-band, the entire region surveyed contains
continuum emission, so no truly suitable `off' position was found within the
survey data.  Similarly, \formaldehyde is ubiquitous across the region, so it
was necessary to `mask out' the absorption lines when building an off position.
This was done by interpolating across the line-containing region with a
polynomial fit (Figure \ref{fig:h2comask}).  The fits were inspected interactively
and tuned to avoid over-predicting the background.

\Figure{figures/a2705.20120915.b0s1g0.00000_offspectra.png}
{An example of the \formaldehyde line masking procedure for building an Off
spectrum.  The line-containing regions for each polarization are shown in cyan
and purple, with the interpolated replacement in red and green.
\todo{Replace this with a nicer version}
}{fig:h2comask}{0.4}{0}

The maps were made by computing an output grid in Galactic coordinates with
15\arcsec pixels and adding each spectrum to the appropriate pixel\footnote{We
use the term `pixel' to refer to a square data element projected on the sky
with axes in Galactic coordinates.  The term `voxel'\todo{, short for
`volumeicture element' (it's not a word)} is used to indicate a cubic data
element, with two axes in galactic coordinates and a third in frequency or
velocity}.  In order
to avoid empty pixels and maximize the signal-to-noise, the spectra were added
to the grid with a weight set from a gaussian with $FWHM=20\arcsec$, which
effectively smooths the output images from $FWHM\approx50\arcsec$ to
$\approx54\arcsec$.  

The Arecibo data were taken at a spectral resolution of 0.68 \kms\footnote{For
most of the map area, data is available at much higher $\lesssim0.2$ \kms
resolution, but the signal-to-noise at this resolution is relatively poor, no
linewidths were observed to be that narrow, and most importantly, one Arecibo
data set suffered from a malfunction that allowed data at 1 \kms resolution to
be taken, but did not acquire the high-resolution data} and the GBT at 0.25
\kms resolution.  The spectra were regridded onto a velocity grid from $-50$ to
150 \kms with 1 \kms resolution.  To achieve this, they were first smoothed by
a gaussian with $FWHM=1 \kms$ then downsampled appropriately.

The Arecibo data were corrected to main beam brightness temperature $T_{MB}$
using a main-beam efficiency $\eta_{MB} = 0.491544 + 0.00580397 za -
0.000341992 za^2$, which is a fit to 5 years worth of calibration data acquired
at Arecibo and assembled by Phil Perrilat.

The Green Bank data have a main beam efficiency $\eta_{MB} = 0.886$, or a gain
of 1.98 K/Jy assuming a 51 \arcsec beam \citep[see][for additional
discussion]{Mangum2013a}.  The GBT data were also corrected for atmospheric
opacity with a typical zenith optical depth $\tau_{z}\approx0.02$, so this
correction was never more than $\sim5\%$.

\section{The Data}

The main products of the mapping data are PPV cubes in the two \formaldehyde
lines, the integrated continuum in the 2 and 6 cm bands, and optical depth data
cubes.  In this section we will describe these data and the systematic errors
associated with them.  In Section \ref{sec:ratiodenscubes} we will examine
ratio maps and cubes and describe the errors associated with them.

\subsection{PPV cubes}
The PPV cubes were created with units of brightness temperature.  The Arecibo
cubes have contributions from 15-20 independent spectra in each pixel, though
this hit rate varies in a systematic striped pattern parallel to the Galactic
plane.  The small overlap regions between different maps have a significantly
higher number of samples; these regions constitute a small portion of the map.
The resulting noise level is RMS $\sim 50-60$ mK except toward the \hii
regions, where it peaks at about 400 mK.  The continuum is derived by averaging
line-free channels; its signal-to-noise peaks at $\sim900$.

The GBT data were mapped in an orthogonal grid pattern, so the hit coverage is
more uniform on small scales, but because of the dual-beam Ku-band system, the
overall noise levels are much more patchy.  Additionally, the nights with
better weather yielded a lower noise level.  The noise ranges from $\sim7$ mK
in the W51 Main region to $\sim 20$ mK in the westmost region.  As with the
Arecibo data, the \hii region adds noise, but the peak noise towards an \hii
region is only $\sim 20$ mK.  This difference is becuase the diffuse \hii
region is fainter at 2 cm.  The signal-to-noise ratio in the continuum peaks at
$\sim 400$.

\subsection{Continuum Images}

While the noise is nominally quite low, there are significant systematic
effects visible in the continuum maps.  The continuum zero-point of each GBT
map was determined by assuming that the first and last scan both observed zero
continuum and that the sky background can be linearly interpolated between the
start and end of the observations.  In general, these are good assumptions, but
they leave in systematic offsets of up to $\lesssim-0.15$ K in the maps, most
likely because there is a $\sim0.15$ K mostly uniform background.

The Arecibo data appear to have smaller systematic offsets, but they are
somewhat more visually pronounced because there is much more diffuse emission
at 6 cm.  The continuum level in the Arecibo data was set to be the 10th
percentile value of each scan, which is effectively the minimum value across
each scan but with added robustness against noise-generated false minima.  In
the eastmost and westmost blocks, this strategy was very effective, as there
are clearly areas in each scan that see no continuum.  However, in the central
block, this approach resulted in a vertical negative filament that almost
certainly represents a local minimum.  This negative filament has values
$\gtrsim-0.08$ K.  Given the excellent agreement between the three
independently observed blocks in their overlap regions, it is clear that the
continuum is reliable above $\gtrsim0.5$ K, which is the entire regime in which
it is a significant contributor to the total background emission (including the
CMB).

\FigureTwoAA{figures/continuum_11.pdf}
            {figures/continuum_22.pdf}
{Continuum images of the 6 cm Arecibo data (left) and 2 cm GBT data (right)}
{fig:continuum}{1.0}{6in}


\subsubsection{Comparison between GBT and GPA data}
\label{sec:gpacompare}
The Galactic Plane ``A'' survey \citep{Langston2000a} covered the Galactic
plane at 14.35 GHz using the NRAO 300 ft telescope, with a reported FWHM beam
size of 6.6\arcmin.  We compared our GBT continuum observations to theirs in
order to determine whether a significant DC component is missing from our data.
Because the GPA used $10\deg$ long scans in Galactic latitude, it should fully
recover all diffuse Galactic Plane emission.  In the released brightness temperature
maps, brightness down to a scale of $1.5\deg$ is recovered.  However, because the
5\arcmin steps between scans is larger than the FWHM of the 14.35 GHz beam,
point source fluxes in the GPA are underestimated by 19\%.

% this analysis is all done in /Users/adam/work/h2co/maps/W51/ipython_log_2014-01-07.py
% then in contcompare_dc_2cm.py
In order to perform the comparison, we first had to correct for an offset
between the GPA and GBT data.  We used the Image Registration toolkit
(\url{http://image-registration.rtfd.org}) to measure the offset between the continuum
images using a cross-correlation technique.
The GBT and Arecibo data match to within 4\arcsec, while the GPA data showed
a 4\arcmin offset in longitude and 1\arcmin in latitude.  We then resampled
the GPA image onto the GBT grid using cubic spline interpolation, then smoothed
both data sets to 9.5\arcmin to account for image artifacts (particularly
vertical streaking) in the GPA data.

We compared the surface brightness in the GPA and GBT data, and found that
the GPA data was $\sim0.2$ K brighter than the GBT in the diffuse portion of
the W51 Main region; the offset is not consistent with a purely multiplicative
offset (Figure \ref{fig:2cmcompare}).  The GBT observed the W51 Main peak to be moderately brighter,
which is likely a result of the sparse sampling in the GPA.  The morphological
agreement between the maps is imperfect, perhaps in part because of the small
area mapped in our GBT data, though there also appears to be vertical (along a
line of constant longitude) stretching of the W51 main source in the unsmoothed
GPA data that is not consistent with the GBT observations.

\Figure{figures/comparison_to_gpa.pdf}
{Comparison between the GBT and NRAO 300 ft \citep{Langston2000a} data.
(top left) NRAO 300 ft 2 cm map
(top right) Arecibo 6 cm map of the same region smoothed to about 8.9\arcmin. 
The colorbar applies to both figures,
showing brightness temperature units in K.  The red contours in both figures
show the region observed by Green Bank; flux outside of those boundaries is
extrapolated.  The green contours show the region where $T_B$(GBT)$>T_B$(300 ft).
(bottom) Plot of the GBT vs the 300 ft surface brightness measurements.
The large red dots show the region within the red contours, while the small
blue dots show the extrapolated points.  
}
{fig:2cmcompare}{0.5}{0}

\subsubsection{Comparison between Arecibo and Urumqi data}
We compare the 6 cm continuum to the Urumqi 25m data from \citet{Sun2007a} and
\citet{Sun2011a}.  Figure \ref{fig:6cmcompare} shows the comparison of the
Urumqi data and the Arecibo continuum data smoothed to 9.5\arcmin resolution.
The agreement is excellent.


\Figure{figures/comparison_to_urumqi_6cm.pdf}
{Comparison between the Arecibo and Urumqi 25m \citep{Sun2011b} data.
(top left) Urumqi 6 cm map of the W51 region.
(top right) Arecibo 6 cm map of the same region smoothed to the 9.5\arcmin
resolution of the Urumqi data set.  The colorbar applies to both figures,
showing brightness temperature units in K.  The red contours in both figures
show the region observed by Arecibo; flux outside of those boundaries is
extrapolated.
(bottom) Plot of the Arecibo vs the Urumqi surface brightness measurements.
The large red dots show the region within the red contours, while the small
blue dots show the extrapolated points.  The dashed gray line is the one-to-one
line.
}
{fig:6cmcompare}{0.5}{0}


\subsubsection{Comparison of GBT and Arecibo data}
In order to compare the Green Bank and Arecibo continuum data, we converted
the brightness temperature maps to Janskys assuming a beam FWHM of 50\arcsec
for both surveys and central frequencies of 4.8 and 14.5 GHz for Arecibo and
Green Bank respectively, however in this section they are referred to as $S_{5
GHz}$ and $S_{15 GHz}$ for brevity.

The data are well-correlated, with $S_{5 GHz} \sim 1.1 S_{15 GHz}$ (Figure
\ref{fig:gbtaocompare}), consistent with a spectral index $\alpha_\nu=-0.1$
typical of optically thin brehmsstrahlung.
However, in Figure \ref{fig:contrrlcompare}a, a great deal of structure in the
$S_{15 GHz}/S_{5 GHz}$ ratio is evident in the vicinity of W51 Main: the ratio
is higher towards the continuum peaks, indicating that the peaks have higher
free-free optical depths than their envelopes.

We additionally compare the radio recombination lines observed simultaneously
with the continuum and \formaldehyde.  Hydrogen RRLs are often extremely
well-correlated with the continuum and are therefore good indicators of the
calibration quality.

In Figure \ref{fig:contrrlcompare}, we show the ratios between the two
frequencies in RRLs and continuum and the line-to-continuum ratios at both
frequencies.  The `line' values are the integrated flux densities over the
range 20 to 100 \kms, which includes all H$\alpha$ emission but no He$\alpha$.

% TODO: re-create h112alpha cube: likely problem with bg subtraction at 20 km/s
% Done!
The ratios between the X and Y axis in each plot in Figure
\ref{fig:contrrlcompare} are fitted using a total least squares approach with
uniform errors for each data point.   The line-to-continuum ratio is
$L/C(H77\alpha)\sim0.06$ and $L/C(H112\alpha)\sim0.03$; in both cases there is
little evidence for deviation from a linear relationship.

\subsubsection{Comparison of the RRL and continuum data}
\label{sec:rrlvscont}
Radio recombination lines are generally observed to be well-correlated with the
corresponding radio continuum, particularly at low frequencies.  At 5 and 15
GHz, the population level departure coefficients are close to 1, $b_n > 0.95$
\citep{Wilson2009a,Walmsley1990a}.

While radio recombination lines are purely thermal in nature, the large-scale
continuum may include a contribution from synchrotron emission.  The
morphological similarity between the 90 cm and 4 m images presented by
\citet{Brogan2013a} and our 6 and 2 cm data hint that synchrotron emission
could be significant.

Figure \ref{fig:gbtaocontcompare} shows a comparison between the integrated RRL
surface brightness and radio continuum at both 2 and 6 cm.  The figure shows
the total least squares best-fit slopes to the data assuming uniform error,
which yield a measurement of the line-to-continuum ratio.

We use the line-to-continuum ratio in both bands to measure the electron
temperature using Equation 14.58 of \citet{Wilson2009a}, which assumes a
plane-parallel, optically-thin emission region with lines formed in local
thermodynamic equilibrium.  The two lines yield consistent measurements, with
mean $T_e\sim7000-8000$ K.  There is only a little structure in the $T_e$ maps,
with a hint of higher temperatures around G49.1-0.4, coincident with the W51C
supernova remnant.



\Figure{figures/continuum_rrl_comparisongrid.pdf}
{Plots of the 5 GHz and 15 GHz continuum and RRL flux densities against one
another; all units are in Jy.  The dashed lines show the total least squares
best fit line with the slope shown in the legend.  Wherever the density of
points is too high to display, the points have been replaced with a contour
plot showing the density of data points.  The upper-right panel shows a
comparison of the continuum ratio to the RRL ratio.  The dashed line in the
upper-right plot has slope 1, and the dotted line has slope 0.6.}
{fig:gbtaocontcompare}{0.5}{0}

\clearpage
\FigureFour
{figures/ratiomap_cont_c2cmtc6cm}
{figures/ratiomap_cont_h77th112}
{figures/ratiomap_cont_h112lc}
{figures/ratiomap_cont_h77lc}
{Ratio maps of W51.  
(a) Continuum ratio $S_{15 GHz} / S_{5 GHz}$.  For $\alpha=-0.1$, an optically
thin free-free source, the ratio is 0.9, while for $\alpha=2$, an optically thick source,
the ratio is 9.
(b) The ratio of the H77$\alpha$ peak to the H112$\alpha$ peak.
(c) The line-to-continuum ratio H112$\alpha$ / $S_{5 GHz}$
(d) The line-to-continuum ratio H77$\alpha$ / $S_{15 GHz}$
}
{fig:contrrlcompare}

\FigureThreePDF
{figures/ratiomap_cont_h112te}
{figures/ratiomap_cont_h77te}
{figures/electron_temperature_77vs111}
{(a) The H112$\alpha$ electron temperature map showing $T_e^*$ in K. 
(b) The H77$\alpha$ electron temperature map showing $T_e^*$ in K.
(c) The measured electron temperature in the 6 cm vs the 2 cm band at each spatial
pixel with significant detected RRL emission.  The contours show regions of
increasing pixel density.  The $x$ marks the median and the $+$ marks the mean
over all valid pixels.
}
{fig:tevste}%{0.5}{0}
\clearpage

\subsection{Optical Depths}

The data cubes were converted into ``optical depth'' data cubes by dividing the
integrated \formaldehyde absorption signature by the measured continuum level.
We added a fixed background of 2.73 K to the reduced images to account for the
CMB, which is absent from the images due to the background-subtraction
process.  We define an ``observer's optical depth''
\begin{equation}
    \tau_{obs} = -\log\left[\frac{T_{mb}}{T_{bg}}\right]
\end{equation}
as opposed to the `true' optical depth, which is modeled in radiative transfer
calculations
\begin{equation}
    \tau = -\log\left[\frac{T_{mb}-T_{ex}}{T_{bg}-T_{ex}}\right]
\end{equation}
The approximation $\tau_{obs} = \tau$ is valid for $T_{ex} << T_{bg}$, which is
true when an \hii region is the backlight but generally not when the CMB is.
Displaying the data on this scale makes regions of similar gas surface density
appear the same, rather than being enhanced where there are backlights.  The
noise is correspondingly suppressed where backlighting sources are present.
%We will examine the effects
%of using $\tau_{obs}$, i.e. assuming $T_{ex} = 0$, in Section \ref{sec:}.


\formaldehyde absorption is ubiquitous across the map.  In the Arecibo data,
8044 of 17800 spatial pixels have peak optical depths $>5\sigma$, and 14547
have peaks $>3\sigma$, so \formaldehyde absorption is detected at $\sim80\%$ of
the observed positions.

The GBT \formaldehyde \twotwo data have lower peak signal-to-noise because the
continuum background is lower.  Additionally, the \twotwo line is expected to
trace denser gas and therefore be detected along fewer lines of sight.  The
\twotwo line is detected with a peak at $>5\sigma$ in 3497 pixels (20\%) and
$>3\sigma$ in 12254 pixels (69\%).  The high detection rate validates
\formaldehyde as an efficient dense-gas tracer.

There were no detections of \formaldehyde \oneone or \twotwo emission.

\subsection{Ratio Cubes}
In order to compare the \formaldehyde spectra, we first
created optical depth ratio cubes.  The data were masked by selecting all
voxels with signal-to-noise $>2$ in  \emph{both} the \oneone and \twotwo data
cubes.  To filter out spurious signals, of which many are expected given the
3.5 million voxels in the data cubes, we filtered the cube with a
number-of-neighbors filter (i.e., a $3\times3$ kernel with all values of unity
except the center pixel) and removed all pixels with fewer than 7 neighbors
with detections.  After filtering, 35039 voxels were left, or $\sim 1\%$ of the
total (though this fraction is arbitrary, since the number of spectral pixels
is determined by the gridding).  These are located in 5822 pixels, or 33\% of
the 17718 pixels valid in both data sets.

Histograms of the data are shown in Figure \ref{fig:datahistograms}.  They
illustrate the non-gaussian nature of the noise.  The ratio distribution, in
particular, appears to follow a power-law distribution with a cutoff at low
ratio.

\Figure{figures/cube_histograms_tau_and_ratio}
{Histograms of the optical depth data cubes.  In panels (a) and (b),
a Gaussian distribution with width equal to the median absolute deviation
is shown for reference; the noise is not well described by a single gaussian
distribution.  In panel (c), the distribution of the ratio is shown.}
{fig:datahistograms}{0.4}{0}


\subsection{Ratio and Density Cubes}
\label{sec:ratiodenscubes}
\todo{[May be removed]}
The \formaldehyde optical depth ratio can be directly transformed into a volume
density of hydrogen $n(\hh)$ by assuming a fixed abundance of \formaldehyde
relative to \hh.  Precise measurements of the \formaldehyde abundance are not
generally available, but typical values of $X_{\ortho} = 10^{-10}-10^{-8}$
relative to \hh are generally assumed \citep{Mangum1993a, Ginsburg2011a,
Ginsburg2013a, Ao2013a} and found to be consistent with the observations.

% ~/work/h2co/codes/ratio_to_dens_plots.py
% ~/repos/h2co_modeling/examples/plot_line_brightness.py
\Figure{figures/tau_ratio_vs_density_varybackground_vary_sigma.pdf}
{The optical depth ratio as a function of \hh density.  The ratio is shown for
$\Xform=10^{-9}$ with two different pairs of background temperature.
The first plot shows both lines in absorption against the CMB, while the second
shows their value for absorption against a bright source with a flat spectral
index, $\alpha\sim0.1$.
The gas temperature is assumed to be 20 K.  The gas is assumed to have a
lognormal density distribution with width $\sigma_s$ \citep[see ][ for
details]{Ginsburg2013a}, so the X-axis indicates the volume-averaged mean \hh
density.  A width $\sigma\sim1$ is typical for turbulence with Mach numbers
$\mathcal{M}\sim1-4$ \citep{Federrath2008a}.  $\sigma=0.01$ is used to
approximate a delta function.}
{fig:ratiovsdens}{0.4}{0}

The curves for converting the optical depth ratio to a density are shown in
Figure \ref{fig:ratiovsdens}.  These curves are generated using RADEX LVG models
\citep[python wrapper \url{https://github.com/keflavich/pyradex/}; original
code][]{van-Der-Tak2007a}.  They assume a velocity gradient of 1 \kms \perpc.
The curves cover the range of plausible \formaldehyde abundances
\citep{Mangum1993a}.  While there are multiple densities corresponding to a
given ratio for $r>8$, we ignore densities $n(\hh)<10^2$ \percc as gas at these
extremely low densities provide negligible optical depth per particle and
therefore should not be detected.

However, the \emph{observed} optical depth is not the true optical depth; it
does not account for the excitation temperature of the molecule.  The curves in
Figure \ref{fig:ratiovsdens} cannot be blindly applied to the observed optical
depth cubes.
\todo{[This means I may remove this section entirely]}


%This is all wrong because it assumes T_ex = 0, which is false.
% The upper limit on the observed optical depth ratio indicates a first concern
% with the data: in Figure \ref{fig:datahistograms}, the ratio distribution
% extends smoothly from $\sim15-40$.  These high ratios are excluded by the LVG
% models, but an examination of the data reveals there is nothing uniquely
% problematic about the regions exhibiting high ratios.  While many of the voxels
% are on cloud edges and therefore potentially affect by signal-to-noise issues
% in the \twotwo line, some of the high-ratio regions are contiguous and have
% high signal-to-noise detections in both lines.  The main region affected by
% this systematic issue is the W51 West region, G49.4-0.31.
% 
% The data provide no ready explanations for these high ratios.  In order to
% be consistent with the models, either the \oneone optical depth must be
% too high or the \twotwo optical depth too low.  This could occur if the
% 6 cm continuum is underestimated or the 2 cm continuum is overestimated.
% The latter possibility is very unlikely, as there are no obvious effects that
% would cause this.  The 6 cm continuum could well be underestimated if there
% is a significant diffuse emission component that is completely subtracted off,
% but this is somewhat implausible.  Comparison to the Sino-German 6 cm continuum
% survey \citep{Sun2007a,Sun2011b} reveals excellent agreement with our data at
% all flux levels (see Figure \ref{fig:6cmcompare}); if anything, the 6 cm
% Arecibo data may slightly overestimate the continuum.
% As discussed in Section \ref{sec:gpacompare}, systematic continuum errors at 2
% cm are $\lesssim 0.2$ K, or $<10\%$, and therefore too small to account for the
% difference.
% 
% The data with `unphysical' optical depth ratios are ignored (i.e., set to NaN)
% when converting the ratio cube to a density cube.
% 
% The density distribution across the whole PPV space is shown in Figure
% \ref{fig:ppvdenshist}.  The density distribution
% is reasonably represented by a lognormal with mean $\log(n(\hh))=3.1$ and
% $\sigma=0.5$, but significant bumps at both lower and higher density were
% notable, so we fit a 3-component model as shown in the figure.
% Lognormal distributions are frequently seen in column density plots
% \citep[e.g.]{Kainulainen2011a,Schneider2013a}, and the bump at high density is
% attributed to a power-law tail generated by gravity.  If such a power-law tail
% is present in our PPV density data, it is very subtle, and a set of 3 clouds
% with different mean densities and similar lognormal density distribution widths
% is an equally acceptable statistical representation of the data.
% 
% The few points with $n>10^{4.5}$ all have highly uncertain densities; their
% densities are likely indeed higher than $10^{4.5}$ \percc, but they may have
% nearly any value above that limit, since the \formaldehyde ratio plateaus to 1
% above that density.


% Wrong =(
% \Figure{figures/cube_histograms_density_ppv.pdf}
% {The density distribution of the PPV cube, following the assumptions outlined
% in Figure \ref{fig:ratiovsdens}, particularly $X(\formaldehyde)=10^{-8.5}$.
% The subplots below show the residuals of the fits to the histograms.
% (left)
% The blue curve shows a composite multi-lognormal fit to the distribution,
% with the three individual components shown in red.
% (right) The red curve shows the best-fit single density component.}
% {fig:ppvdenshist}{0.4}{0}

% \subsubsection{Model Uncertainty}
% There is a wide acceptable range of \formaldehyde abundances and \hh density
% distributions to be considered.  Figure \ref{fig:ppvdenshistmulti} shows the
% density distribution derived using two different abundances and two different
% lognormal density distribution widths.  Using a $\sim30\times$ (1.5 dex) lower
% abundance results in a $\sim3\times$ (0.5 dex) higher density, demonstrating
% that our density measurements are fairly robust against even large abundance
% variations.  Changes in the density distribution can have similar effects on
% the inferred mean density; the $\delta$-function distribution shown (which is
% entirely implausible) results in a similar $\sim3\times$ increase in the mean
% density, but a factor of $\sim2$ narrowing in the density distribution.
% 
% \Figure{figures/cube_histograms_density_ppv_multimodel.pdf}
% {Density distributions from the PPV density cube as shown in Figure
% \ref{fig:ppvdenshist}, but for different assumptions about the \formaldehyde
% abundance and lognormal density distribution width.  $\sigma=0$ indicates a
% $\delta$-function distribution.}
% {fig:ppvdenshistmulti}{0.4}{0}
% 
% 
% \subsection{Projections}
% Projections along the velocity axis - peak, sum, or other mathematical
% operations - create simple 2-dimensional images that can be readily compared to
% other data sets and simulations.  Figure \ref{fig:densprojections} shows the
% mean and peak projections along the density data cube.  
% 
% One feature of particularly low mean and peak density appears at $\ell=49.5,
% -0.3<b<-0.2$, while another elongated structure at $48.8<\ell<49, b=-0.3$
% exhibits an overall uniform high density.  The W51 Main region
% ($\ell=49.5$,$b=-0.4$) also shows relatively high densities towards its center,
% but with evidence of a radial decrease in density to its outskirts.


%% projection_figures.py
%\FigureTwoAA{figures/density_mean_projection_withhist.pdf}
%            {figures/density_peak_projection_withhist.pdf}
%{Projections of the derived density along with histograms
%showing the density distribution and best-fit lognormal.
%(top) Projected mean density.
%(bottom) Projected peak density.
%Two main high-density regions are evident: W51 Main at 49.5 -0.4 and G48.9-0.3 at 48.9 -0.3.}
%{fig:densprojections}{1.0}{6in}

% We show a projection of the velocity at peak density in each pixel (Figure
% \ref{fig:velopv}).  This figure clearly highlights the 68 \kms cloud as a
% high-density feature throughout the region \citep{Carpenter1998a}.  However,
% there are significant gaps in the molecular gas, notably a $\sim6 pc$ gap
% (assuming $D=5$ kpc) at $\ell=49.1, b=-0.34$.
% 
% \FigureTwoAA
% {figures/velocity_at_peak_density.pdf}
% {figures/pvdiagram_max_along_latitude.pdf}
% {(top) The velocity at peak density.
%  (bottom) A position-velocity diagram showing the peak density across latitude
%  at each longitude.  There is a density enhancement at the highest velocities
%  across the entire $\sim100$ pc slice; it may be an indication of an impinging 
%  spiral density wave (discussion to be added later XXXX).
% }
% {fig:velopv}{1.0}{6in}

% The position-velocity diagram in Figure \ref{fig:velopv} shows the same general
% feature: the highest velocity gas is the densest.  There appears to be an edge
% of high-density gas tracing the most redshifted velocity at each longitude.
% Given the $\sim100$ pc extent of the cloud and the high-density ridge, it is
% entirely implausible that this structure is caused by any local phenomenon,
% e.g. supernovae or cloud collisions.  Instead, the high-density ridge provides
% evidence for a spiral density wave triggering the high densities.

\section{Structure of the Molecular Clod [sic]}
The \formaldehyde observations reveal two essential features of the W51 GMC:
its density structure and its line-of-sight geometry.

\subsection{Model fitting and geometry}
Both \formaldehyde lines are seen only in absorption.  However, in some cases
the absorption is against a continuum background, while in others the
absorption may be only against the CMB.

We have fit the \formaldehyde lines constrained by the LVG models (Section
\ref{sec:models}) to spectra averaged over regions with coherent molecular
absorption signatures.  We compared the $\chi^2$ values for fits with the
observed continuum as the background to those with the background fixed to
$T_{BG} = T_{CMB}$.  We then selected the better of the two fits as
representative of the physical conditions.

It is possible that there are multiple continuum emitters along the line of
sight in many cases, with the absorbing molecular gas somewhere in the middle.
While this possibility adds uncertainty to the measurements, there are some
cases in which the dominant continuum can unambiguously be assigned a
foreground or background position.

%\FigureSVG{figures/geometry_diagram.pdf_tex}
%{A sketched diagram of the W51 region}
%{fig:geosketch}{6cm}
\Figure{figures/geometry_diagram.png}
{A sketched diagram of the W51 region as viewed from the Galactic north pole,
with the observer looking up the page from the bottom (i.e., W51C is the 
front-most labeled object along our line-of-sight).
\todo{This figure needs to be kept up to date with text changes, and interaction
between the cloud components should be shown more clearly.  Clouds need to be labeled
by their coordinates and velocity components as well}}
{fig:geosketch}{0.5}{0}

\subsection{Kinematic Maps}
\todo{Add maps of peak-velocity-per-pixel, separated into components if
possible}

\FigureTwo
{figures/H2CO11_central_velocity}
{figures/H110a_central_velocity}
{(left) Velocity of the peak \formaldehyde \oneone signal (deepest absorption) at
1 \kms resolution
(right) Velocity of the peak H110$\alpha$ emission as derived from gaussian fits
to each spectrum.}
{fig:kinematics}{1}

\subsection{W51 B}
\label{sec:w51b}
The W51 B filament is rich in molecular gas, but has relatively weak
\formaldehyde absorption.  The absorption models are inconsistent with
the molecular gas being in front of the continuum emission.
Figure \ref{fig:h2cofrontbackmodel} shows an example model fit with the
continuum assumed to be in front and in back, illustrating that the best-fit
model parameters with continuum in the back do not reproduce the data.

% pyspeckit_individual_fits
% paperfigure_filament_demonstrate_frontback(fixedTO=True):
\Figure{figures/spectralfits_70kmscloud_aperture_ap6_modelcomparison_withresiduals.pdf}
{An example of the difference in models between a continuum source (red) and
the CMB (green) as the background.  The top plot shows the \oneone line and the
bottom shows the \twotwo line both with the continuum level set to zero in the
plot but treated as a frozen parameter in the fit.  The residuals are shown
offset above the spectra, with the dashed line indicating the zero-residual
level.  The grey shaded regions show the 1-$\sigma$ error bars on each pixel.
The model with the CMB as the only
background is able to reproduce the absorption line, while the model with the
\hii region in the background cannot account for the depth of the \twotwo line.
The reduced $\chi^2/n$ for the models are 14.1 (red) and 2.8 (green), evaluated
only over the pixels where the model is greater than the local RMS.}
{fig:h2cofrontbackmodel}{0.5}{0}

% The mass-weighted mean density ranges from $\sim$1.5\ee{4} to 5.6\ee{4} \percc,
% with the lowest mean density corresponding to the highest column density toward
% the center.

The relative positioning of the molecular gas behind the \hii regions suggests
that they are also behind the W51 C supernova remnant.

\subsection{The edge of W51 C}
W51 C is a supernova remnant that spatially overlaps with the W51 B star
forming region.  \citet{Brogan2013a} argue that the supernova remnant must be
in front of the \hii region G49.2-0.3 because the \hii-region has not absorbed
all of the 4m nonthermal emission.  The G49.1-0.4, G49.0-0.3, and G48.9-0.3
regions, however, show absorption signatures and may be in the foreground.
There are clumps aligned along the 68 \kms filamentary cloud with very high CO
and HI velocities \citep{Koo1997b,Koo1997c,Brogan2013a}, indicating that the
SNR is interacting with the molecular gas.

The clumps at $\sim68$ \kms are either lower density ($n<1.5\ee{4}$ \percc) and in the
background of the HII region or high density ($n>1.5\ee{5}$ \percc), low-column
density and in the foreground.  The $62$ \kms clumps have densities a few times
higher, $n\sim4\ee{4}$ \percc, and are clearly in the foreground of the
continuum emission because their absorption depths are $\sim2.5$ K, which
cannot occur for absorption against the CMB.  Figure \ref{fig:contbetween63kms}
shows a model spectrum fitted assuming the continuum lies between the two
molecular velocity components.  The relative strength of the \thirteenco and
the \formaldehyde also suggests that the 68 \kms component is behind the
continuum.

We are seeing molecular gas both in front of and behind the supernova.  This
geometry can be readily confirmed by looking for molecular absorption at much
lower frequencies where the SN synchrotron emission dominates over the \hii
region free-free emission, i.e. the 335 and 71 MHz \para lines.

\Figure{figures/spectralfits_63kmscloud_aperture_ap3_both_legend.pdf}
{The spectrum extracted from G49.119-0.277, showing a model in which the continuum
is \emph{behind} the 63 \kms component but in front of the 68 \kms component.}
{fig:contbetween63kms}{0.5}{0}

\subsection{The 66 \kms IRDC}
Between W51 A and W51 B, there is a component of the 68 \kms cloud that is
filamentary and in the foreground of all of the free-free emission.
This cloud component is evident as an IRDC in the Spitzer GLIMPSE images from
$\ell=49.393 b=-0.357$ to $\ell=49.207 b=-0.338$.

The \hii region G49.20-0.35 is clearly behind the IRDC, though there are strong
morphological hints that it is interacting with and truncated by the cloud.

\Figure{figures/filaments_H2CO_13CO_pvslice.pdf}
{A position-velocity slice of the 68 \kms cloud, which includes an infrared dark
cloud and the interaction region with the W51C supernova remnant.
(top) \formaldehyde \oneone observed optical depth
(middle) \formaldehyde \twotwo observed optical depth
(bottom) \thirteenco 1-0 emission from the GRS with \formaldehyde \oneone
contours superposed.  The weakness of the \formaldehyde absorption on the right
half of the cloud corroborates the geometry inferred from comparison of the \oneone
and \twotwo lines in Figure \ref{fig:h2cofrontbackmodel}.}
{fig:filament_pvslice}{0.5}{0}

\subsubsection{G49.27-0.34}
% digging_in_to_G49.27-0.34.py
The \uchii region G49.27-0.34, which was considered a candidate extended green
object (EGO) and subsequently rejected for lack of \hh emission
\citep{deBuizer2010a,Lee2013a}, exhibits a second velocity component at $\sim68
\kms$, slightly but clearly redshifted of the rest of the IRDC.  Its mass is
$\sim2\ee{3} \msun$ based on the BGPS flux and using the assumptions outlined
in \citet{Aguirre2011a}, suggesting that the high velocity could be due to
infall or virialized gas within a deep potential.  The virialized velocity
width, given the radius and mass from the BGPS data, is $\sigma_{vir}=8.8 \kms$.

Both radio continuum and RRLs are detected toward this source.  The H77$\alpha$
RRL velocity is $\sim60$ \kms, significantly blueshifted from the molecular
gas.  The \formaldehyde lines do not independently distinguish between the
continuum source being in the front or back of the cloud, but the mean density
from the BGPS mass and radius $n\sim2.5\ee{4}$ \percc is within a factor of 2
of the \formaldehyde-derived density, $n\sim1.4\ee{4}$ \percc, if the continuum
source is behind the gas, while the \formaldehyde-derived density is too low,
$n\sim2\ee{3}$ \percc if the continuum source is in front.

The implied geometry therefore has the \hii region behind the molecular gas,
plowing toward it at a velocity difference $\Delta v \sim 8$ \kms.
Such a high velocity difference may indicate that the \hii region is confined by
the molecular gas and on a plunging orbit into the cloud.
% Notes to self:
% Considered the alternative: what if the cloud was actually much hotter and
% therefore lower mass?  Then, the HII region could be in front.  That would
% also explain the CO self-absorption.  However, the temperatures required are
% ~50-100 K, but the CO temperature is ~15K
% Could do an SED fit to check; that is not worth the effort at present

\subsubsection{G49.4-0.3f, aka G49.34-0.34}
% 70kmscloudLeft ap1
The \hii region centered at 49.34-0.34 was identified by \citet{Mehringer1994a}
as part of the G49.4-0.3 complex.  There are 3 distinct \formaldehyde line
components at 51, 63.70, and 68.47 \kms.  The 51 \kms component is behind the \hii
region; the \thirteenco line is detected at comparable brightness at 51 \kms
and 63 \kms, while the \formaldehyde \oneone line is $\sim10\times$ deeper at
63 \kms.  The RRLs associated with this source are at $v_{LSR}=58 \pm 1$ \kms.

The \formaldehyde lines are moderately well-fit by the two-velocity-component
model, but there is a relative excess of \twotwo absorption at 66 \kms.  The
extra absorption may indicate that there is a high-density, low-column
component at this velocity.

The 8 \um GLIMPSE image shows that the 68 \kms IRDC crosses in front of this
source.  Herschel Hi-Gal 70\um images reveal a ring structure that is hinted at
in the 8 \um image.  There is no evidence for interaction between the ring
feature and the IRDC.  This intriguing feature will likely be difficult to
study in detail because the dusty, molecular gas feature lies in front of it.

\subsection{G49.4-0.3}
\label{sec:maus}
The collection of \hii regions around G49.4-0.3 vaguely resembles a cartoon
mouse.  The molecular gas in this region is separated into two distinct
components, one at 51 \kms and the other at 64 \kms.  The 64 \kms component is
in the foreground, while the 51 \kms is in the background of most of the \hii
regions.  

Both cloud components are in the foreground of the central \hii regions at
G49.36-0.31, the `eyes' of the mouse.  The density of the 51 \kms component is
an order of magnitude higher than that in the 64 \kms component in this region,
suggesting that the gas is being compressed by the \hii region.

The clean separation between the 64 and 51 \kms cloud components suggests that
they are not interacting at this location.  

\subsection{Infrared Dark Cloud G49.47-0.27}
The cloud to the north of W51 Main/IRS2 appears as a dark feature in Spitzer
GLIMPSE 8 \um maps.  It is detected in \formaldehyde from 54 to 64
\kms.  Throughout, it has a high \oneone/\twotwo ratio, $\gtrsim7$ in
most voxels, indicating a low density $n\lesssim10^3$ \percc.  The \oneone
optical depth is high, up to $\sim1/3$, so if there is any clumping, the gas
may be optically thick, which would imply that our density measurement is an
overestimate.

Centered at 60.6 \kms, the region has a line FWHM 5 - 7 \kms, indicating that
it is quite turbulent, with 3D Mach number in the range $10 < \mathcal{M} < 20$
for an assumed $10 < T < 20$ K.  At its centroid velocity, it is connected
to the rest of the W51 GMC.

There is a bubble HII region in the north part of this cloud, at G49.470-0.255,
with radius $0.016\deg$ (1.5 pc).  The \hii region is at a $v_{lsr}=50$\kms.
Because it is not detected in Brackett $\gamma$ emission (Bally et al?
UKIDSS?), it is most likely behind the cloud.

The \citet{Kang2009a} Spitzer survey of YSOs in the region indicates that there
are no YSOs within the boundaries of this cloud; it is very likely
non-star-forming at present.

\subsection{HII region G49.37-0.30, AKA W51 West}
W51 West is a busy and luminous \hii region; it is the second-brightest radio
and millimeter continuum source in our survey after the W51 Main/IRS2 region.
As noted in \citet{Carpenter1998a}, its velocity ($v_{LSR}\sim 50\kms$ XXX) 
may indicate that the cloud is unassociated with the W51 Main GMC.


\section{A formation scenario for the W51 region}
The W51 cloud complex has been discussed as both a collection of unrelated
clouds and a tight complex of interacting clouds \citep{Carpenter1998a,
Kang2010a}.  The \formaldehyde data presented here support the idea that the 68
\kms cloud is a coherent entity and that it is interacting with other clouds
associated with W51 A.  The presence of W51 Main at the interaction point
between multiple clouds hints that its great mass and star-forming potential
was triggered by this cloud-cloud collision.  It is likely that both observed
clouds were pre-existing features in a larger, possibly atomic, cloud that
underwent stretching upon interaction with a spiral arm.


\section{Molecular Gas and \formaldehyde modeling}
\label{sec:h2co}

Figure \ref{fig:peakoptdepth} shows the most important observed properties of
the \formaldehyde lines.  The figures show the peak observed optical depth
$\tau_{obs} = -\log(T_{MB}/T_{continuum})$ in each line along with the ratio of
the \oneone to the \twotwo optical depth.  They are masked to show significant
pixels determined by:
\begin{enumerate}
    \item Selecting all voxels with $S/N > 2$ and with at least 7 (of 26
        possible) neighbors also having $S/N > 2$
    \item Selecting all voxels with $>=10$ neighbors having $S/N > 2$
    \item Growing the included mask region by 1 pixel in all directions
    \item Selecting all voxels with $>=5$ neighbors already selected
    \item When used to mask 2D images, the selection is then collapsed such
        that any pixel containing at least one voxel along the line of sight is
        included
\end{enumerate}
This approach effectively includes all significant pixels and all reliably
detected regions within the data cube, though the number of neighbors used at
each step and the selected growth size are somewhat arbitrary and could be
modified with little effect.

Figure \ref{fig:peakoptdepth} contains two ratio maps.  The first shows the
observed optical depth ratio, while the second shows the `true' optical depth
ratio assuming an excitation temperature for each line, $T_{ex}(\oneone) = 1.0$
K and $T_{ex}(\twotwo) = 1.5$ K.  These excitation temperatures are
representative of those expected for most of the modeled parameter space in
which absorption is expected.  Fitting of individual lines-of-sight confirm
that good fits can be achieved using these assumed temperatures.
Figure \ref{fig:ratiovsdens} can be applied to Figure \ref{fig:peakoptdepth}d
with reasonable accuracy, while applying it to Figure \ref{fig:peakoptdepth}c
would yield incorrect results.

However, there are some clear outliers within the map: the clouds at G48.9-0.3
and G49.4-0.2 both show very low \oneone/\twotwo ratios over a broad area.  As
discussed in Sections \ref{sec:w51b} and \ref{sec:maus}, these two regions have
\hii regions in the foreground of the molecular gas.  The ratios displayed in
Figure \ref{fig:peakoptdepth} are therefore computed with an incorrect
background assumed.

\subsubsection{High density gas in G49.5-0.4?}
\label{sec:g495}
The region G49.47-0.42 also appears to have a very low ratio.  In this region,
there are two deep lines of \formaldehyde \oneone, both indicating absorption
against an \hii region.  Curiously, toward this region, the deeper \twotwo line
corresponds to the shallower \oneone line, so the ratio
is nonphysical.  However, when comparing the optical depth of the two line
components directly (Figure \ref{fig:lowerpeakoptdepth}), the deeper \twotwo
component remains peculiar: the simple RADEX LVG models cannot accomodate this
deeper \twotwo line with any combination of density, temperature, column
density, and ortho-to-para ratio.  There are a few possible explanations for 
this difficulty:
\begin{enumerate}
    \item The \twotwo line is overestimated because of calibration issues
    \item There is a significant emission component affecting the \oneone
        line, making it appear weaker (shallower) than it really is
    \item Multiple density and/or velocity components are present
    \item Multiple continuum sources exist along the line of sight, one
        optically thin dominating the 6 cm emission and the other optically
        thick, dominating the 2 cm emission.
\end{enumerate}
Option (1) is implausible, since the corresponding line at high velocity (68
\kms) shows no peculiarities.  Option (2) is possible, though extant VLA
observations seem to rule it out, and there is a very narrow range of parameter
space in which \oneone emission would occur simultaneous with \twotwo
absorption, especially given the bright ($\sim22$ K) background at 6 cm.  After
testing many potential combinations of parameters, we have ruled out option
(3).  Option (4) is seemingly possible, although the required spectral index
of the 6 cm source is $\alpha\sim -2$, which is a much steeper inverse spectral
index than is ever observed elsewhere.  Option (4) points to the possibility
that there is some systematic error in the continuum of perhaps the 6 cm data,
there is no hint of such an issue in the RRL comparison in Section
\ref{sec:rrlvscont}.

\FigureFourPDF
{figures/peak_observed_opticaldepth_11}
{figures/peak_observed_opticaldepth_22}
{figures/peak_observed_opticaldepth_ratio}
{figures/peak_observed_opticaldepth_ratio_tex}
{Plots of the peak \emph{observed} optical depth $\tau_{obs} =
-\log(T_{MB}/T_{continuum})$ in the (a) \oneone and (b) \twotwo lines and (c)
their ratio, \oneone / \twotwo.  Figure (d) shows the `true' optical depth ratio
assuming $T_{ex}(\oneone) = 1.0$ K and $T_{ex}(\twotwo) = 1.5$ K; these are
reasonable and representative excitation temperatures but they are not fits to
the data.
The data are masked with a filter described in Section \ref{sec:h2co} and cover
the range $75 > V_{LSR} > 40$ \kms; see Figures \ref{fig:lowerpeakoptdepth} and
\ref{fig:upperpeakoptdepth} for individual velocity components.  In general,
lower (redder) ratios in figures (c) and (d) indicate higher densities, however
in the filament at 49.0-0.3, the low ratio is due to the geometry in which
$T_{continuum}$ is in the \emph{foreground} of the molecular gas.}
{fig:peakoptdepth}


\FigureFourPDF
{figures/lowerpeak_observed_opticaldepth_11}
{figures/lowerpeak_observed_opticaldepth_22}
{figures/lowerpeak_observed_opticaldepth_ratio}
{figures/lowerpeak_observed_opticaldepth_ratio_tex}
{Same as Figure \ref{fig:peakoptdepth}, but limited to $62 > V_{LSR} > 40$ \kms.}
{fig:lowerpeakoptdepth}

\FigureFourPDF
{figures/upperpeak_observed_opticaldepth_11}
{figures/upperpeak_observed_opticaldepth_22}
{figures/upperpeak_observed_opticaldepth_ratio}
{figures/upperpeak_observed_opticaldepth_ratio_tex}
{Same as Figure \ref{fig:peakoptdepth}, but limited to $75 > V_{LSR} > 62$ \kms.}
{fig:upperpeakoptdepth}

\todo{Despite everything, I made density maps.  About 10 times with different
values.  The net lesson is almost always the same.}

\FigureThreePDF
{figures/W51_H2CO_logdensity_textbg_max_ratio_sigma0.5}
{figures/W51_H2CO_logdensity_textbg_mid_ratio_sigma0.5}
{figures/W51_H2CO_logdensity_textbg_min_ratio_sigma0.5}
{Density maps of the W51 cloud complex.  The density maps are created by
mapping the (a) maximum (b) median (c) minimum observed optical depth ratio to
density using the RADEX models with the measured background temperature.  The
models used assume $\Xform = 10^{-9}$ and $\sigma_s = 0.5$; a range of model
choices is valid but they all yield systematic shifts.  These figures show that
a moderate-density component ($n(\hh)\sim3000$ \percc) is present in at least
one spectral bin at each position, while some regions have peak densities
exceeding $n>10^6$ \percc.
We accounted for the relative line-of-sight location of the continuum sources
by setting the background to 2.73 K within the G48.9-0.3 and G49.4-0.2 regions.
We did not attempt to correct the continuum in locations where the continuum
source is between two clouds.
The apparently high peak density of the G49.5-0.4 region is discussed in greater
detail in \ref{sec:g495}.
}
{fig:densitymaps}

% Describe fitting models to absorption lines (NOT to tau)
% 
% A local cloud at $v_{lsr}\sim5 \kms$ is detected in \formaldehyde \oneone
% across most of the cloud and not detected at \twotwo, with
% $\tau_{\oneone}/\tau_{\twotwo} \gtrsim 3$, implying a very low column
% $N_{\formaldehyde}\sim10^{11.5}$ or $N_{\hh} \sim 10^{20.5}$.  The cloud is
% seen in \thirteenco as a very weak, diffuse feature, and in HI absorption as a
% narrow (self)-absorption feature.

%% Pixel fitting doesn't look as good as the above approaches.  Let's stick with
%% them.
% \subsection{Pixel Fitting}
% We fit the data with \formaldehyde models on a pixel-by-pixel basis.  Nearly
% all pixels with significant detections include two blended velocity components.
% Two-component fits are never particularly stable, so it was necessary to
% restrict the parameters being fitted.
% 
% The model fitting provides a complementary view to the optical depth ratio maps
% shown in Figure \ref{fig:peakoptdepth}.  Regions of lower signal to noise
% appear scattered in Figure \ref{fig:w51h2cofits}, but each pixel contains
% information about the velocity, velocity width, column (\perkms \perpc), and
% density.
% 
% 
% \FigureTwo{figures/W51_H2CO_2parfittry10_v1_densityvelocity.pdf}
%           {figures/W51_H2CO_2parfittry10_v2_densityvelocity.pdf}
% {Density and velocity fits to the \formaldehyde 
% data cubes.  The left figure shows velocity components restricted
% to have central $V_{LSR} < 65$ \kms, while the right figure shows
% velocity components with $V_{LSR} > 65$ \kms.  
% %The yellow regions in the top panel correspond to \oneone
% %detections and \twotwo nondetections, indicating upper limits $n<10^{3.8}$
% %(68\% confidence) or $n<10^{4.3}$ (99.7\% confidence).
% }
% {fig:w51h2cofits}{1}

\section{Conclusion}
We have presented maps of the \formaldehyde \oneone and \twotwo and H77$\alpha$
and H111$\alpha$ lines covering the W51 star forming complex.


 
% \FigureTwo{figures/MCMC_DensColplot_67_64.png}{figures/spec67_64_bestfit_MCMC.png}
% {Plots demonstrating upper limit fits.  The left plot shows the allowed
% parameter space from MCMC sampling of the data given the RADEX model.  The
% right plot shows the `best-fit' model to the optical depth spectra, which is
% clearly unconstrained by the relatively insensitive \twotwo\ spectrum.  The
% sensitivity in the \oneone line is better in large part because of brighter 6
% cm background across the whole W51 region.  Despite the lack of constraint on the
% volume density, there is a reasonably strong constraint on the column density.}
% {fig:w51MCMCcompare}{1}
% 
% The molecular gas is concentrated near, but not exactly on, the bright cm
% peaks.  W51 IRS2 has a massive clump of gas at 65 \kms, and W51 e2 has a
% similar clump.  However, e2 also seems to have a very dense ($n>10^5 \percc$)
% infalling clump.  The spectra, along with multicomponent fits, are shown in
% Figure \ref{fig:w51hiispectra}.
% 
% \FigureTwo{figures/W51_bestfit_spec53_49_IRS2.png}{figures/W51_bestfit_spec53_49_W51e2.png}
% {Plots of the optical depth spectra centered on W51 IRS2 (left) and W51 e2, an
% ultracompact HII region (right).  IRS2 shows high-density gas with a slight
% hint of infall, but otherwise a somewhat vanilla spectrum.  W51e2 has a large,
% high-density red shoulder, indicating high-density gas at the most red velocity
% in the system.  Because this is foreground gas, that high-density gas
% \emph{must} be moving towards the \uchii region.}
% {fig:w51hiispectra}{1}

\textbf{Acknowledgements}:
We thank Xiaohui Sun for providing the Urumqi 6 cm Stokes I image prior to its
availability on the survey website.

\appendix
\section{A complete record of the data}
In the interest of reproducibility, we include a complete code suite detailing
all of the steps from raw data to final reduced product.  If one acquires all
of the raw data from both Arecibo and GBT, the data and derivative products
from this publication can all be reproduced in their entirety.

The Arecibo data was processed using code based on the aoIDL package by Phil
Perillat.  The \texttt{accum\_map} routine performs the standard on-off reduction
steps and stores the resulting time-series of spectra in a FITS file.  The
\texttt{make\_off} routine creates an `off' spectrum by averaging selected
(ideally emission-free) scans, smoothing them, and interpolating across
regions of the spectrum with lines or electronic artifacts.  The data were then
gridded into map cubes using the \texttt{makecube} routines in \texttt{sdpy}.

The GBT data was processed using exclusively custom-made scripts stored in
\texttt{sdpy}.  The KFPA pipeline
(\url{https://safe.nrao.edu/wiki/bin/view/Kbandfpa/ObserverGuide}) is a useful
alternative, but at the time these data were taken, it was not quite mature
enough for use with our Ku-band data.  The custom-made scripts also provided
needed flexibility for comparing the GBT and Arecibo data.


Dependencies:
\begin{itemize}
    \item aoIDL \url{http://www.naic.edu/~phil/download/aoIdl.tar.gz}
    \item sdpy \url{https://github.com/keflavich/sdpy}
\end{itemize}


/Users/adam/work/h2co/maps/paper/solobib.tex
\end{document}

