/Users/adam/work/h2co/maps/paper/preface.tex

\section{Introduction}


\section{Observations \& Data Reduction}

(observations paragraph)
The W51 survey was completed in September 2011.  
Data were taken in on-the-fly mode with the GBT Ku-band dual-beam system
and the Arecibo C-band receiver.  On the GBT, cross-hatched north-south
and east-west maps were created in Galactic coordinates, while at Arecibo
only east-west maps were taken.

The data was reduced using custom-made scripts based off of both GBTIDL's mapping
routines (cite Langston?) and Phil Perillat's AOIDL routines.  The code is available
at \url{https://github.com/keflavich/gbtpy}.

The individual spectral were calibrated using a noise calibration diode as
usual.  For the GBT data, the first and last scans of each observation were
used as off positions, and the background level to be subtracted off of the
continuum was determined by linearly interpolating between these scans.

The Arecibo data reduction process for W51
presented unique challenges: at C-band, the entire region surveyed contains
continuum emission, so no truly suitable `off' position was found within the
survey data.  Similarly, \formaldehyde is ubiquitous across the region, so it
was necessary to `mask out' the absorption lines when building an off position.
This was done by interpolating across the line-containing region with a
polynomial fit (Figure \ref{fig:h2comask}).  The fits were inspected interactively
and tuned to avoid over-predicting the background.

\Figure{figures/a2705.20120915.b0s1g0.00000_offspectra.png}
{An example of the \formaldehyde line masking procedure for building an Off
spectrum.  The line-containing regions for each polarization are shown in cyan
and purple, with the interpolated replacement in red and green.
}{fig:h2comask}{0.4}{0}

The maps were made by computing an output grid in Galactic coordinates with
15\arcsec pixels and adding each spectrum to the appropriate pixel\footnote{We
use the term `pixel' to refer to a square data element projected on the sky
with axes in Galactic coordinates.  The term `voxel' is used to indicate a
cubic data element, with two axes in galactic coordinates and a third in
frequency or velocity}.  In order
to avoid empty pixels and maximize the signal-to-noise, the spectra were added
to the grid with a weight set from a gaussian with $FWHM=20\arcsec$, which
effectively smooths the output images from $FWHM\approx50\arcsec$ to
$\approx54\arcsec$.  

The Arecibo data were taken at a spectral resolution of 0.68 \kms\footnote{For
most of the map area, data is available at much higher $\lesssim0.2$ \kms
resolution, but the signal-to-noise at this resolution is relatively poor, no
linewidths were observed to be that narrow, and most importantly, one Arecibo
data set suffered from a malfunction that allowed data at 1 \kms resolution to
be taken, but did not acquire the high-resolution data} and the GBT at 0.25
\kms resolution.  The spectra were regridded onto a velocity grid from $-50$ to
150 \kms with 1 \kms resolution.  To achieve this, they were first smoothed by
a gaussian with $FWHM=1 \kms$ then downsampled appropriately.

\section{The Data}

The main products of the mapping data are PPV cubes in the two \formaldehyde
lines, the integrated continuum in the 2 and 6 cm bands, and optical depth data
cubes.  In this section we will describe these data and the systematic errors
associated with them.  In section XXX we will examine ratio maps and cubes and
describe the errors associated with them.

\subsection{PPV cubes}
The PPV cubes were created with units of brightness temperature.  The Arecibo
cubes have contributions from 15-20 independent spectra in each pixel, though
this hit rate varies in a systematic striped pattern parallel to the Galactic
plane.  The small overlap regions between different maps have a significantly
higher number of samples; these regions constitute a small portion of the map.
The resulting noise level is RMS $\sim 50-60$ mK except toward the \hii
regions, where it peaks at about 400 mK.  The continuum is derived by averaging
line-free channels; its signal-to-noise peaks at $\sim900$.

The GBT data were mapped in an orthogonal grid pattern, so the hit coverage is
more uniform on small scales, but because of the dual-beam Ku-band system, the
overall noise levels are much more patchy.  Additionally, the nights with
better weather yielded a lower noise level.  The noise ranges from $\sim7$ mK
in the W51 Main region to $\sim 20$ mK in the westmost region.  As with the
Arecibo data, the \hii region adds noise, but the peak noise towards an \hii
region is only $\sim 20$ mK.  This difference is becuase the diffuse \hii
region is fainter at 2 cm.  The signal-to-noise ratio in the continuum peaks at
$\sim 400$.

\subsection{Continuum Images}

While the noise is nominally quite low, there are significant systematic
effects visible in the continuum maps.  The continuum zero-point of each GBT
map was determined by assuming that the first and last scan both observed zero
continuum and that the sky background can be linearly interpolated between the
start and end of the observations.  In general, these are good assumptions, but
they leave in systematic offsets of up to $\lesssim-0.15$ K in the maps, most
likely because there is a $\sim0.15$ K mostly uniform background.

The Arecibo data appear to have smaller systematic offsets, but they are
somewhat more visually pronounced because there is much more diffuse emission
at 6 cm.  The continuum level in the Arecibo data was set to be the 10th
percentile value of each scan, which is effectively the minimum value across
each scan but with added robustness against noise-generated false minima.  In
the eastmost and westmost blocks, this strategy was very effective, as there
are clearly areas in each scan that see no continuum.  However, in the central
block, this approach resulted in a vertical negative filament that almost
certainly represents a local minimum.  This negative filament has values
$\gtrsim-0.08$ K.  Given the excellent agreement between the three
independently observed blocks in their overlap regions, it is clear that the
continuum is reliable above $\gtrsim0.5$ K, which is the entire regime in which
it is a significant contributor to the total background emission (including the
CMB).

\FigureTwoAA{figures/continuum_11.pdf}
          {figures/continuum_22.pdf}
{Continuum images of the 6 cm Arecibo data (left) and 2 cm GBT data (right)}
{fig:continuum}{1.0}{6in}


\subsubsection{Comparison to GPA data}
\label{sec:gpacompare}
The Galactic Plane ``A'' survey \citep{Langston2000a} covered the Galactic
plane at 14.35 GHz using the NRAO 300 ft telescope, with a reported FWHM beam
size of 6.6\arcmin.  We compared our GBT continuum observations to theirs in
order to determine whether a significant DC component is missing from our data.
Because the GPA used $10\deg$ long scans in Galactic latitude, it should fully
recover all diffuse Galactic Plane emission.

% this analysis is all done in /Users/adam/work/h2co/maps/W51/ipython_log_2014-01-07.py
In order to perform the comparison, we first had to correct for an offset
between the GPA and GBT data.  We used the Image Registration toolkit
(\url{image-registration.rtfd.org}) to measure the offset between the continuum
images using a cross-correlation technique.
The GBT and Arecibo data match to within 4\arcsec, while the GPA data showed
a 4 arcminute offset in longitude and 1 arcminute in latitude.  We then resampled
the GPA image onto the GBT grid using cubic spline interpolation, then smoothed
the GBT data to 8.9\arcmin (instead of 6.6\arcmin; because there were significant
patterned residuals when smoothing to only 6.6\arcmin, we performed a $\chi^2$ fit
to determine the best smoothing FWHM).

We then compared the surface brightness in the GPA and GBT data, and found that
the GPA data was $\sim0.2$ K brighter than the GBT in the W51 Main region and
about equal to the west.  The morphological agreement between the maps is
imperfect, perhaps in part because of the small area mapped in our GBT data,
though there also appears to be vertical (along a line of constant longitude)
stretching of the W51 main source in the GPA data that is not consistent with
the GBT observations.


\subsection{Optical Depths}

The data cubes were converted into ``optical depth'' data cubes by dividing the
integrated \formaldehyde absorption signature by the measured continuum level.
We added a fixed background of 2.73 K to the reduced images to account for the
CMB, which is absent from the images due to the background-subtraction
process.

Because of the varying background level, the optical depth cubes have highly
variable signal-to-noise.  In the Arecibo data, 8044 of 17800 pixels have peak
optical depths $>5\sigma$, and 14547 have peaks $>3\sigma$, so \formaldehyde
absorption is detected at $\sim80\%$ of the observed positions.

The GBT \formaldehyde \twotwo data have lower peak signal-to-noise due to lower
continuum background.  Additionally, the \twotwo line is expected to trace
denser gas and therefore detected on fewer lines of sight.  The \twotwo line is
detected with a peak at $>5\sigma$ in 3497 pixels (20\%) and $>3\sigma$ in
12254 pixels (69\%).  The high detection rate validates \formaldehyde as a
highly efficient dense-gas tracer.

\subsection{Ratio Cubes}
In order to use the \formaldehyde ratio to determine densities, we first
created optical depth ratio cubes.  The data were masked by selecting all
voxels with signal-to-noise $>2$ in  \emph{both} the \oneone and \twotwo data
cubes.  To filter out spurious signals, of which many are expected given the
3.5 million voxels in the data cubes, we filtered the cube with a
number-of-neighbors filter (i.e., a $3\times3$ kernel with all values of unity
except the center pixel) and removed all pixels with fewer than 7 neighbors
with detections.  After filtering, 35039 voxels were left, or $\sim 1\%$ of the
total.  These are located in 5822 pixels, or 33\% of the 17718 pixels valid in
both data sets.

Histograms of the data are shown in Figure \ref{fig:datahistograms}.  They
illustrate the non-gaussian nature of the noise.  The ratio distribution, in
particular, appears to follow a power-law distribution with a cutoff at low
ratio.

\Figure{figures/cube_histograms_tau_and_ratio}
{Histograms of the optical depth data cubes.  In panels (a) and (b),
a Gaussian distribution with width equal to the median absolute deviation
is shown for reference; the noise is not well described by a single gaussian
distribution.  In panel (c), the distribution of the ratios is shown.
Values of the ratio $\gtrsim15$ are nonphysical, but these are only observed
in a small fraction of the pixels}
{fig:datahistograms}{0.4}{0}


\subsection{Ratio and Density Cubes}
The \formaldehyde ratio can be directly transformed into a volume density of
hydrogen $n(\hh)$ by assuming a fixed abundance of \formaldehyde relative to
\hh.  Precise measurements of the \formaldehyde abundance are not generally
available, but typical values of $X_{\ortho} = 10^{-10}-10^{-8}$ relative to
\hh are generally assumed \citep{Mangum1993a, Ginsburg2011a, Ginsburg2013a,
Ao2013a} and found to be consistent with the observations.

\Figure{figures/tau_ratio_vs_density_thinlimit_t20_sigma1.pdf}
{The optical depth ratio as a function of \hh density.  The ratio is shown for
a range of \formaldehyde abundances as indicated in the legend.
The gas temperature is assumed to be 20 K, and the \formaldehyde lines
are assumed to both be optically thin.  The gas is assumed to have a
lognormal density distribution with width $\sigma=1.0$ \citep[see ][ for
details]{Ginsburg2013a}, so the X-axis indicates the volume-averaged
mean \hh density.  This width is typical for turbulence with Mach numbers
$\mathcal{M}\sim1-4$ \citep{Federrath2008a}.}
{fig:ratiovsdens}{0.4}{0}

The curves for converting the optical depth ratio to a density are shown in
Figure \ref{fig:ratiovsdens}.  These curves are generated using RADEX LVG models
\citep[python wrapper \url{https://github.com/keflavich/pyradex/}; original
code][]{van-Der-Tak2007a}.  They assume a velocity gradient of 1 \kms \perpc.
The curves cover the range of plausible \formaldehyde abundances
\citep{Mangum1993a}.  While there are multiple densities corresponding to a
given ratio for $r>8$, we ignore densities $n(\hh)<10^2$ \percc as gas at these
extremely low densities provide negligible optical depth per particle and
therefore should not be detected.

The upper limit on the observed optical depth ratio indicates a first concern
with the data: in Figure \ref{fig:datahistograms}, the ratio distribution
extends smoothly from $\sim15-40$.  These high ratios are excluded by the LVG
models, but an examination of the data reveals there is nothing uniquely
problematic about the regions exhibiting high ratios.  While many of the voxels
are on cloud edges and therefore potentially affect by signal-to-noise issues
in the \twotwo line, some of the high-ratio regions are contiguous and have
high signal-to-noise detections in both lines.  The main region affected by
this systematic issue is the W51 West region, G49.4-0.31.

The data provide no ready explanations for these high ratios.  In order to
be consistent with the models, either the \oneone optical depth must be
too high or the \twotwo optical depth too low.  This could occur if the
6 cm continuum is underestimated or the 2 cm continuum is overestimated.
The latter possibility is very unlikely, as there are no obvious effects that
would cause this.  The 6 cm continuum could well be underestimated if there
is a significant diffuse emission component that is completely subtracted off,
but this is somewhat implausible.  Comparison to the Sino-German 6 cm continuum
survey \citep{Sun2011b} .... XXX ....
As discussed in Section \ref{sec:gpacompare}, systematic continuum errors at 2
cm are $\lesssim 0.2$ K, or $<10\%$, and therefore too small to account for the
difference.

The data with `unphysical' optical depth ratios are ignored (i.e., set to NaN)
when converting the ratio cube to a density cube.

The density distribution across the whole PPV space is shown in Figure
\ref{fig:ppvdenshist}.  The density distribution
is reasonably represented by a lognormal with mean $\log(n(\hh))=3.1$ and
$\sigma=0.5$, but significant bumps at both lower and higher density were
notable, so we fit a 3-component model as shown in the figure.
Lognormal distributions are frequently seen in column density plots
\citep[e.g.]{Kainulainen2011a,Schneider2013a}, and the bump at high density is
attributed to a power-law tail generated by gravity.  If such a power-law tail
is present in our PPV density data, it is very subtle, and a set of 3 clouds
with different mean densities and similar lognormal density distribution widths
is an equally acceptable statistical representation of the data.

The few points with $n>10^{4.5}$ all have highly uncertain densities; their
densities are likely indeed higher than $10^{4.5}$ \percc, but they may have
nearly any value above that limit, since the \formaldehyde ratio plateaus to 1
above that density.


\Figure{figures/cube_histograms_density_ppv.pdf}
{The density distribution of the PPV cube, following the assumptions outlined
in Figure \ref{fig:ratiovsdens}, particularly $X(\formaldehyde)=10^{-8.5}$.
The subplots below show the residuals of the fits to the histograms.
(left)
The blue curve shows a composite multi-lognormal fit to the distribution,
with the three individual components shown in red.
(right) The red curve shows the best-fit single density component.}
{fig:ppvdenshist}{0.4}{0}

\subsubsection{Model Uncertainty}
There is a wide acceptable range of \formaldehyde abundances and \hh density
distributions to be considered.  Figure \ref{fig:ppvdenshistmulti} shows the
density distribution derived using two different abundances and two different
lognormal density distribution widths.  Using a $\sim30\times$ (1.5 dex) lower
abundance results in a $\sim3\times$ (0.5 dex) higher density, demonstrating
that our density measurements are fairly robust against even large abundance
variations.  Changes in the density distribution can have similar effects on
the inferred mean density; the $\delta$-function distribution shown (which is
entirely implausible) results in a similar $\sim3\times$ increase in the mean
density, but a factor of $\sim2$ narrowing in the density distribution.

\Figure{figures/cube_histograms_density_ppv_multimodel.pdf}
{Density distributions from the PPV density cube as shown in Figure
\ref{fig:ppvdenshist}, but for different assumptions about the \formaldehyde
abundance and lognormal density distribution width.  $\sigma=0$ indicates a
$\delta$-function distribution.}
{fig:ppvdenshistmulti}{0.4}{0}


\subsection{Projections}
Projections along the velocity axis - peak, sum, or other mathematical
operations - create simple 2-dimensional images that can be readily compared to
other data sets and simulations.  Figure \ref{fig:densprojections} shows the
mean and peak projections along the density data cube.  

One feature of particularly low mean and peak density appears at $\ell=49.5,
-0.3<b<-0.2$, while another elongated structure at $48.8<\ell<49, b=-0.3$
exhibits an overall uniform high density.  The W51 Main region
($\ell=49.5$,$b=-0.4$) also shows relatively high densities towards its center,
but with evidence of a radial decrease in density to its outskirts.


% projection_figures.py
\FigureTwoAA{figures/density_mean_projection_withhist.pdf}
            {figures/density_peak_projection_withhist.pdf}
{Projections of the derived density along with histograms
showing the density distribution and best-fit lognormal.
(top) Projected mean density.
(bottom) Projected peak density.
Two main high-density regions are evident: W51 Main at 49.5 -0.4 and G48.9-0.3 at 48.9 -0.3.}
{fig:densprojections}{1.0}{6in}

We also show a projection of the velocity at peak density in each pixel (Figure
\ref{fig:velopv}).  This figure clearly highlights the 68 \kms cloud as a
high-density feature throughout the region \citep{Carpenter1998a}.  However,
there are significant gaps in the molecular gas, notably a $\sim6 pc$ gap
(assuming $D=5$ kpc) at $\ell=49.1, b=-0.34$.

\FigureTwoAA
{figures/velocity_at_peak_density.pdf}
{figures/pvdiagram_max_along_latitude.pdf}
{(top) The velocity at peak density.
 (bottom) A position-velocity diagram showing the peak density across latitude
 at each longitude.
}
{fig:velopv}{1.0}{6in}

The position-velocity diagram in Figure \ref{fig:velopv} shows the same general
feature: the highest velocity gas is the densest.  There appears to be an edge
of high-density gas tracing the most redshifted velocity at each longitude.
Given the $\sim100$ pc extent of the cloud and the high-density ridge, it is
entirely implausible that this structure is caused by any local phenomenon,
e.g. supernovae or cloud collisions.  Instead, the high-density ridge provides
evidence for a spiral density wave triggering the high densities.

\section{Structure of the Molecular Clod}




% \subsection{Model Fitting}
% 
% These $\tau$ cubes are then fit with the RADEX models used for other
% \formaldehyde fitting.  However, there are multiple velocity components in W51,
% so I used a two-component (unconstrained) fit for each pixel, which is
% frequently unstable but in the case of W51 looks to have produced reasonable
% results.  Note that there was \emph{no} \formaldehyde emission detected anywhere
% in the W51 region.
% 
% A first interesting note is that a local cloud at $v_{lsr}\sim5 \kms$ is
% detected in \formaldehyde \oneone across most of the cloud and not detected at
% \twotwo, with $\tau_{\oneone}/\tau_{\twotwo} \gtrsim 3$, implying a
% very low column
% $N_{\formaldehyde}\sim10^{11.5}$ or $N_{\hh} \sim 10^{20.5}$.  
% The cloud is seen in \thirteenco as a very weak, diffuse feature, and in HI absorption
% as a very sharp, deep (self)-absorption feature.
% % This density measurement
% % is consistent with observations from \citet{Ginsburg2011} of high density in
% % GMCs.  However, GMCs are generally thought of as being low-density clouds, so
% % this result may be surprising.
% 
% %FIGURE: mcmc column vs density 
% 
% I successfully made density maps of the W51 cloud, though because the velocity
% structure is quite complicated, it was necessary to fit two components to most of the
% map.  Two-component fits are never particularly stable, so it was necessary to
% restrict the parameters being fitted, and even then the results aren't
% perfectly reliable.  Despite those caveats, there are some reliable fits,
% particularly towards the `core' of W51 Main / W51 IRS 2.  There are two
% high-density components with $n\sim10^5-10^{5.5}$ \percc at different velocities evident
% in Figure \ref{fig:w51h2cofits}.  The southern component, centered on W51 Main,
% has $v_{LSR}\sim56-59$.  The northern component, a strip going through IRS 2
% and towards the west, peaks around $v_{LSR}\sim68-69$ \kms.  A 10 \kms difference
% between two extremely dense components, both which are necessarily in the
% foreground of the HII region, is shocking (probably, anyway, unless the sound
% speed is very high).
% 
% 
% \FigureTwo{figures/W51_H2CO_2parfit_v1_densityvelocity.png}
% {figures/W51_H2CO_2parfit_v2_densityvelocity.png}
% {Density and velocity fits to the W51 Arecibo and GBT \formaldehyde 
% data cubes.  The yellow regions in the top panel correspond to \oneone
% detections and \twotwo nondetections, indicating upper limits $n<10^{3.8}$
% (68\% confidence) or $n<10^{4.3}$ (99.7\% confidence).}
% {fig:w51h2cofits}{1}
% 
% There is a large area where \oneone was detected, but \twotwo was not.  Our
% sensitivity allows us to place a modest upper limit on the gas density, with
% $3-\sigma$ upper limits $\lesssim10^{4.3}$ \percc (but the most likely
% densities are $10^2 < n < 10^4$ \percc).  Figure \ref{fig:w51MCMCcompare} shows
% a particular model for a spectrum that is especially unconstrained.  The
% \oneone/\twotwo optical depth ratio in this object is $\sim10-20$, indicating that
% the volume density must be low.
% 
% \FigureTwo{figures/MCMC_DensColplot_67_64.png}{figures/spec67_64_bestfit_MCMC.png}
% {Plots demonstrating upper limit fits.  The left plot shows the allowed
% parameter space from MCMC sampling of the data given the RADEX model.  The
% right plot shows the `best-fit' model to the optical depth spectra, which is
% clearly unconstrained by the relatively insensitive \twotwo\ spectrum.  The
% sensitivity in the \oneone line is better in large part because of brighter 6
% cm background across the whole W51 region.  Despite the lack of constraint on the
% volume density, there is a reasonably strong constraint on the column density.}
% {fig:w51MCMCcompare}{1}
% 
% The molecular gas is concentrated near, but not exactly on, the bright cm
% peaks.  W51 IRS2 has a massive clump of gas at 65 \kms, and W51 e2 has a
% similar clump.  However, e2 also seems to have a very dense ($n>10^5 \percc$)
% infalling clump.  The spectra, along with multicomponent fits, are shown in
% Figure \ref{fig:w51hiispectra}.
% 
% \FigureTwo{figures/W51_bestfit_spec53_49_IRS2.png}{figures/W51_bestfit_spec53_49_W51e2.png}
% {Plots of the optical depth spectra centered on W51 IRS2 (left) and W51 e2, an
% ultracompact HII region (right).  IRS2 shows high-density gas with a slight
% hint of infall, but otherwise a somewhat vanilla spectrum.  W51e2 has a large,
% high-density red shoulder, indicating high-density gas at the most red velocity
% in the system.  Because this is foreground gas, that high-density gas
% \emph{must} be moving towards the \uchii region.}
% {fig:w51hiispectra}{1}

\appendix{A complete record of the data}
In the interest of reproducibility, we include a complete code suite detailing
all of the steps from raw data to final reduced product.

The Arecibo data was first processed using the aoIDL package by Phil Perillat.

The GBT data was processed using exclusively custom-made scripts stored in
gbtpy.  The KFPA pipeline (Langston) is a useful alternative, but at the time
these data were taken, it was not quite mature enough for use with our Ku-band
data.  The custom-made scripts also provided needed flexibility for comparing
the GBT and Arecibo data.

Dependencies:
\begin{itemize}
    \item aoIDL \url{http://www.naic.edu/~phil/download/aoIdl.tar.gz}
    \item gbtpy \url{https://github.com/keflavich/gbtpy}
\end{itemize}

/Users/adam/work/h2co/maps/paper/solobib.tex
\end{document}

