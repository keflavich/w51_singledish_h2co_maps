/Users/adam/work/h2co/maps/paper/preface.tex

\title{The W51 GMC in 3D: The DGMF}
% NEED A BETTER TITLE...
\newcommand{\casa}{2}
\newcommand{\eso}{1}
\newcommand{\cfa}{3}
\newcommand{\yale}{4}

\author{Adam Ginsburg\altaffilmark{\eso},
        John Bally\altaffilmark{\casa},
        Cara Battersby\altaffilmark{\cfa},
        Jeremy Darling\altaffilmark{\casa},
        Allison Youngblood\altaffilmark{\casa},
        Mayra Santos\altaffilmark{\yale},
        Hector Arce\altaffilmark{\yale},
        and probably a few others (Roberto Galvan-Madrid, Jonathan Tan)
        }
\email{Adam.G.Ginsburg@gmail.com}

\affil{{$^\casa$}{\it{CASA, University of Colorado, 389-UCB, Boulder, CO 80309}}}
\affil{{$^\eso$}{\it{European Southern Observatory, Karl-Schwarzschild-Strasse 2, D-85748 Garching bei München, Germany}}}
\affil{{$^\monash$}{\it{Monash Centre for Astrophysics, School of Mathematical Sciences, Monash University, Vic 3800, Australia}}}
\affil{{$^\cfa$}{\it{CfA}}}
\affil{{$^\yale$}{\it{Yale}}}


%\begin{centering} 
%    \today
%\end{centering}



% Somewhere in here:
% need a shoutout to "Overheard on Astro-PH"
%
% FOCUS ON THE *OBSERVATIONAL* QUESTIONS solved here... the theoretical just
% aren't going to be that exciting

\section{Abstract}
We present new 2 cm and 6 cm maps of \formaldehyde, radio recombination lines,
and the radio continuum in the W51 star forming complex acquired with Arecibo
and the Green Bank Telescope at $\sim50\arcsec$ resolution.
We perform a detailed analysis of the data and data processing with an emphasis on
ensuring the reliability of the data.
The data set has been made public at \url{http://dx.doi.org/10.7910/DVN/26818}.

We present an analysis of the three-dimensional structure of the W51 region,
determining the relative line-of-sight positions of molecular and ionized gas.
We show evidence that the W51 C supernova remnant is interacting with the W51 B
molecular filament.
We argue that the W51 A and B star-forming regions are physically associated and
interacting, supporting the idea that the W51 A cluster formation was triggered
by cloud-cloud collision.

We measure gas densities using the \formaldehyde densitometer, finding largely
consistent mass-weighted volume densities $n\gtrsim10^4$ \percc throughout the
W51 GMC, with higher density $n\gtrsim10^5$ \percc gas associated with the
proto-clusters.  We present continuous measurements of the dense gas mass
fraction over the range $10^4$ $\percc< n(\hh) < 10^6$ \percc.

We did not detect \emph{any} \formaldehyde emission throughout the W51 GMC; all
gas dense enough to emit under normal conditions is in front of bright
continuum sources.  This nondetection implies that the \formaldehyde emission
detected in other galaxies, e.g. Arp 220, comes from high-density gas that is
not directly affiliated with forming massive stars, i.e. the entire ISM of
these galaxies is very dense.


\footnote{Compiled on \today\ at \currenttime}


\section{Introduction}
\todo{To do.  To-do items are coded in red.}

Massive star clusters are among the most visually outstanding features in the
night sky.  In other galaxies, they are useful probes of the star formation history
and can be individually identified and measured \citep{Bastian2008a}.  Locally,
they are the essential laboratories in which we can study the formation of massive
stars \citep{Davies2012a}.

In order to utilize these clusters as laboratories, we need to understand their
formation in detail.  Clusters are often assumed and measured to be coeval
\citep[e.g.][]{Kudryavtseva2012a}, but uncertainties remain \citep{Beccari2010a}.
In the most massive clusters, there are predictions that multiple generations
or an extended generation of stars should form prior to gas expulsion because
the gas will remain gravitationally bound \citep{Bressert2012a}.   Feedback from
and within young massive clusters is an active field of numerical study
\citep{Rogers2013a,Dale2013a,Dale2012a,Dale2008a,Dale2005a,Parker2013a,Myers2014a,Krumholz2014a}
While
only 5-35\% of all stars form in bound clusters \citep{Kruijssen2012a},
these clusters form the basis of our understanding of stars and stellar evolution
[citation needed], and understanding their formation is therefore crucial.

The results of cluster formation may be decided before the first stars are
formed.  The starless initial conditions of massive clusters have not yet been
definitively observed \citep{Ginsburg2012a} though there are viable candidates
such as G0.253+0.016 \citep{Longmore2012b}.  The initial conditions for star
formation on any scale are clearly turbulent.  However, there is no evidence
whether these initial conditions differ in any qualitative way from turbulence
in local, low-mass star-forming regions.

\subsection{W51}
The W51 cloud complex (Figure \ref{fig:w51large}), containing the two W51
massive proto-cluster candidates from \citep{Ginsburg2012a}, is located at
$\ell\sim49, b\sim-0.3$, very near the Galactic midplane\footnote{The midplane
at $d=5.1$ kpc is offset approximately -0.22 to -0.33 degrees from $b=0$
depending on our solar system's height above the midplane, $Z_\odot = 20$ pc or
30 pc, respectively \citep{Reed2006a,Joshi2007a}.} at a distance of 5.1 kpc
\citep{Sato2010a}.  It is a well-known and thoroughly studied collection of
clouds massing $M>10^6 \msun$
\citep{Carpenter1998a,Bieging2010a,Kang2010a,Parsons2012a}.  The radio-bright
regions are generally known as W51 A to the east, W51 B to the west, and W51 C
for the southern component, known to trace a supernova remnant
\citep{Koo1995a,Brogan2000a,Brogan2013a}.

\Figure{figures/W51_wisecolor_modified_largescale_labeled.pdf}
{A color composite of the W51 region with major regions, W51 A, B, and C,
labeled.  The blue, green, and red colors are WISE bands 1, 3, and 4 (3.4, 12,
and 22 \um) respectively.  The yellow-orange semitransparent layer is from the
Bolocam 1.1 mm Galactic Plane Survey data \citep{Aguirre2011a,Ginsburg2013a}.
Finally, the faint whitish haze filling in most of the image is from a 90 cm
VLA image by \citet{Brogan2013a}, which primarily traces the W51 C supernova
remnant.}
{fig:w51large}{0.5}{0}


\subsection{Formaldehyde}
Formaldehyde (\formaldehyde) has been recognized as a useful probe of physical
conditions in the molecular interstellar medium for decades
\citep{Mangum1993a}.  The centimeter lines, \formaldehyde \oneone (6.2 cm,
4.82966 GHz) and \twotwo (2.1 cm, 14.48848 GHz), have a peculiar excitation
process in which collisions overpopulate the lower of the two $K_c$ rotational
states.  Because the \oneone and \twotwo level pairs populate differently
depending on the volume density of the colliding partner (generally \hh), their
ratio is sensitive to the local gas volume density.

\formaldehyde \oneone has been observed in the W51 main region with the VLA
\citep{Martin-Pintado1985a} and Westerbork \citep{Arnal1985a}, and this data
was used to gain some early constraints on the geometry of the region
\citep[e.g.][]{Carpenter1998a}.  \citet{Henkel1980a} presented observations of
the \oneone and \twotwo lines, and \citet{Martin-Pintado1985b} presented
single-dish mapping observations of the \formaldehyde \twotwo line toward the
W51 Main region, but both treated the region as a single-density structure.



\section{Observations \& Data Reduction}

\subsection{GBT 2 cm}
The W51 survey was performed in September 2011 and 2012.  The GBT data were
taken as part of program AGBT10B/019; the raw data are available from the NRAO
archive (\url{https://archive.nrao.edu/}).  The data presented in this paper
include sessions 10, 11, 14, 16, 17, 20, 21, and 22; the other sessions from
this project include maps of outer galaxy regions and a single-pointing survey
of Galactic plane sources that will be presented in another paper. 

Data were taken in on-the-fly mode with the GBT Ku-band dual-beam system.
Cross-hatched north-south and east-west maps were created in Galactic
coordinates. 

The data was reduced using custom-made scripts based off of both GBTIDL's
mapping routines by Glen Langston
(\url{https://safe.nrao.edu/wiki/bin/view/Kbandfpa/KfpaPipelineHowTo}) and Phil
Perillat's AOIDL routines.  The code is available at
\url{https://github.com/keflavich/sdpy}.  The data reduction code and workflow
is included in a corresponding git repository:
\url{https://github.com/keflavich/w51_singledish_h2co_maps}.

The individual spectral were calibrated using a noise calibration diode as
usual.  The first and last scans of each observation were used as off
positions, and the background level to be subtracted off of the continuum was
determined by linearly interpolating between these scans.

The GBT data infrequently exhibited major data artifacts; a key phase of the
reduction process was excising bad observations, which were usually isolated to
a single component of the backend.

The Green Bank data have a main beam efficiency $\eta_{MB} = 0.886$, or a gain
of 1.98 K/Jy assuming a 51 \arcsec beam \citep[see][for additional
discussion]{Mangum2013a}.  The GBT data were also corrected for atmospheric
opacity using Ron Maddalena's
\texttt{getForecastValues}\footnote{\url{http://www.gb.nrao.edu/~rmaddale/Weather/}}
with a typical zenith optical depth $\tau_{z}\approx0.02$, so this
correction was never more than $\sim5\%$.

Typical noise levels were $\sim10-20$ mK per 1 \kms channel; the levels vary
across the map.  See Section \ref{sec:ppvcubes} for details.

\subsection{Arecibo 6 cm}
The Arecibo data were taken as part of project A2705 over the course of 4
nights, September 10, 11, 12, and 15 2012.  On the first night, September 10
2012, a significant fraction of the data was lost due to an internal instrument
error within the Mock spectrometer, which resulted in a loss of the
high-resolution component of the \formaldehyde data for that night.  As a
result, we have focused our study on the lower-resolution ($\sim1$ \kms) data.

The fields were observed with east-west maps using the C-band receiver.  No
crosshatching was performed with Arecibo.

The Arecibo data reduction process for W51
presented unique challenges: at C-band, the entire region surveyed contains
continuum emission, so no truly suitable `off' position was found within the
survey data.  Similarly, \formaldehyde is ubiquitous across the region, so it
was necessary to `mask out' the absorption lines when building an off position.
This was done by interpolating across the line-containing region with a
polynomial fit.  The fits were inspected interactively
and tuned to avoid over-predicting the background.

%  (Figure \ref{fig:h2comask})
% \Figure{figures/a2705.20120915.b0s1g0.00000_offspectra.png}
% {An example of the \formaldehyde line masking procedure for building an Off
% spectrum.  The line-containing regions for each polarization are shown in cyan
% and purple, with the interpolated replacement in red and green.
% \todo{Replace this with a nicer version}
% }{fig:h2comask}{0.4}{0}

The Arecibo data were corrected to main beam brightness temperature $T_{MB}$
using a main-beam efficiency as a function of zenith angle in degrees ($za$):
$$\eta_{MB}(za) = 0.491544 + 0.00580397 za - 0.000341992 za^2$$
This is a fit to 5 years worth of calibration data acquired at Arecibo and
assembled by Phil Perrilat.

Typical noise levels were $\sim50$ mK per 1 \kms channel.  See Section
\ref{sec:ppvcubes} for details.

\subsection{Mapmaking}
The maps were made by computing an output grid in Galactic coordinates with
15\arcsec pixels and adding each spectrum to the appropriate pixel\footnote{We
use the term `pixel' to refer to a square data element projected on the sky
with axes in Galactic coordinates.  The term `voxel'\todo{, short for
`volumeicture element' (it's not a word)} is used to indicate a cubic data
element, with two axes in galactic coordinates and a third in frequency or
velocity}.  In order
to avoid empty pixels and maximize the signal-to-noise, the spectra were added
to the grid with a weight set from a Gaussian with $FWHM=20\arcsec$, which
effectively smooths the output images from $FWHM\approx50\arcsec$ to
$\approx54\arcsec$.  See \citet{Mangum2007a} for more detail on the on-the-fly
mapping technique used here.

The Arecibo data were taken at a spectral resolution\footnote{For
most of the map area, data is available at much higher $\lesssim0.2$ \kms
resolution, but the signal-to-noise at this resolution is relatively poor, no
linewidths were observed to be that narrow, and most importantly, one Arecibo
data set suffered from a malfunction that allowed data at 0.68 \kms resolution
to be taken, but not the corresponding 0.2 \kms data.} of 0.68 \kms and the GBT
at 0.25 \kms resolution.  The spectra were regridded onto a velocity grid from
$-50$ to 150 \kms with 1 \kms resolution.  To achieve this, they were first
smoothed by a Gaussian with $FWHM=1$ \kms then downsampled appropriately.


\section{The Data}

The main products of the mapping data are PPV cubes in the two \formaldehyde
lines, the integrated continuum in the 2 and 6 cm bands, and optical depth data
cubes.  In this section we describe these data and the systematic errors
associated with them.  

\subsection{PPV cubes}
\label{sec:ppvcubes}
The PPV cubes were created with units of brightness temperature.  The Arecibo
cubes have contributions from 15-20 independent spectra in each pixel, though
this hit rate varies in a systematic striped pattern parallel to the Galactic
plane.  The small overlap regions between different maps have a significantly
higher number of samples; these regions constitute a small portion of the map.
The resulting noise level is RMS $\sim 50-60$ mK except toward the \hii
regions, where it peaks at about 400 mK.  The continuum is derived by averaging
line-free channels; its signal-to-noise peaks at $\sim900$.

The GBT data were mapped in an orthogonal grid pattern, so the hit coverage is
more uniform on small scales, but because of the dual-beam Ku-band system, the
overall noise levels are much more patchy.  Additionally, the nights with
better weather yielded a lower noise level.  The noise ranges from $\sim7$ mK
in the W51 Main region to $\sim 20$ mK in the westmost region.  As with the
Arecibo data, the \hii region adds noise, but the peak noise towards an \hii
region is only $\sim 20$ mK.  This difference is because the diffuse \hii
region is fainter at 2 cm.  The signal-to-noise ratio in the continuum peaks at
$\sim 400$.

\subsection{Continuum Images}
Our observations comprise the highest resolution wide-area maps of the W51
region in the 6 cm and 2 cm continuum that preserve large angular scale
structures.  They can be used to provide the zero-spacing for future and extant
VLA observations and are essential for the density measurements we describe
later.  We therefore carefully verify the quality and calibration of the
continuum data.

While the noise is nominally quite low, there are significant systematic
effects visible in the continuum maps.  The continuum zero-point of each GBT
map was determined by assuming that the first and last scan both observed zero
continuum and that the sky background can be linearly interpolated between the
start and end of the observations.  In general, these are good assumptions, but
they leave in systematic offsets of up to $\lesssim-0.15$ K in the maps, most
likely because there is a $\sim0.15$ K mostly uniform background.

The Arecibo data appear to have smaller systematic offsets, but they are more
visually pronounced because there is much more diffuse emission
at 6 cm.  The continuum zero-point level in the Arecibo data was set to be the
10th percentile value of each scan, which is effectively the minimum value
across each scan but with added robustness against noise-generated false
minima.  In the eastmost and westmost blocks, this strategy was very effective,
as there are clearly areas in each scan that see no continuum.  However, in the
central block, this approach resulted in a vertical negative filament that
almost certainly represents a local minimum that should be positive.  This
negative filament has values $\gtrsim-0.08$ K.  Given the excellent agreement
between the three independently observed blocks in the overlap regions between
these blocks, it is clear that the continuum is reliable above $\gtrsim0.5$ K,
which is the entire regime in which it is a significant contributor to the
total background emission (at lower levels, the CMB is dominant).

\FigureTwoAA{figures/continuum_11.pdf}
            {figures/continuum_22.pdf}
{Continuum images of the 6 cm Arecibo data (left) and 2 cm GBT data (right)}
{fig:continuum}{1.0}{6in}


\subsubsection{Comparison between GBT and GPA data}
\label{sec:gpacompare}
The Galactic Plane ``A'' survey \citep{Langston2000a} covered the Galactic
plane at 14.35 GHz using the NRAO 300 ft telescope, with a reported FWHM beam
size of 6.6\arcmin.  We compared our GBT continuum observations to theirs in
order to determine whether a significant DC component is missing from our data.
Because the GPA used $10\deg$ long scans in Galactic latitude, it should fully
recover all diffuse Galactic Plane emission.  In the released brightness temperature
maps, brightness down to a scale of $1.5\deg$ is recovered.  However, because
the GPA data undersampled the sky (its 5\arcmin steps between scans were larger
than the FWHM of the 14.35 GHz beam), point source fluxes in the GPA are
underestimated by 19\% and flux on small angular scales may be unreliable.

% this analysis is all done in /Users/adam/work/h2co/maps/W51/ipython_log_2014-01-07.py
% then in contcompare_dc_2cm.py
In order to perform the comparison, we first had to correct for an offset
between the GPA and GBT data.  We used the Image Registration toolkit
(\url{http://image-registration.rtfd.org}) to measure the offset between the continuum
images using a cross-correlation technique.
The GBT and Arecibo data match to within 4\arcsec, while the GPA data showed
a 4\arcmin offset in longitude and 1\arcmin in latitude.  We then resampled
the GPA image onto the GBT grid using cubic spline interpolation, then smoothed
both data sets to 9.5\arcmin.  There are image artifacts (particularly
vertical streaking) in the GPA data that are diminished by this large
smoothing kernel.

We compared the surface brightness in the GPA and GBT data, and found that the
GPA data was $\sim0.2$ K brighter than the GBT in the diffuse portion of the
W51 Main region; the offset is not consistent with a purely multiplicative
offset (Figure \ref{fig:2cmcompare}).  The GBT observed the W51 Main peak to be
moderately brighter, which is likely a result of the sparse sampling in the
GPA.  The morphological agreement between the maps is imperfect, perhaps in
part because of the small area mapped in our GBT data, though there also
appears to be vertical (along a line of constant longitude) stretching of the
W51 main source in the unsmoothed GPA data that is not consistent with the GBT
observations.

\Figure{figures/comparison_to_gpa.pdf}
{Comparison between the GBT and NRAO 300 ft \citep{Langston2000a} data.
(top left) NRAO 300 ft 2 cm map
(top right) Arecibo 6 cm map of the same region smoothed to about 8.9\arcmin. 
The colorbar applies to both figures,
showing brightness temperature units in K.  The red contours in both figures
show the region observed by Green Bank; flux outside of those boundaries is
extrapolated.  The green contours show the region where $T_B$(GBT)$>T_B$(300 ft).
(bottom) Plot of the GBT vs the 300 ft surface brightness measurements.
The large red dots show the region within the red contours.  
}
{fig:2cmcompare}{0.5}{0}

\subsubsection{Comparison between Arecibo and Urumqi data}
We compare the 6 cm continuum to the Urumqi 25m data from \citet{Sun2007a} and
\citet{Sun2011a}.  Figure \ref{fig:6cmcompare} shows the comparison of the
Urumqi data and the Arecibo continuum data smoothed to 9.5\arcmin resolution.
The Arecibo data appears to be systematically brighter than the Urumqi data
by about 40\%.  This systematic offset is worrisome for spectral index
measurements, but should have little effect on the \formaldehyde analysis.
\todo{This offset appeared upon reanalysis in July 2014.  It definitely, 100\%
certainly, did not exist the last time I ran the \emph{exact} same analysis
on the same data with the same code.  I cannot track down this offset but it
will probably keep me awake at nights for months.  The increase from before
is about 25\%, which one will note is nowhere near $1/\eta_{MB}=2$, so
I really have no explanation.}


\Figure{figures/comparison_to_urumqi_6cm.pdf}
{Comparison between the Arecibo and Urumqi 25m \citep{Sun2011b} data.
(top left) Urumqi 6 cm map of the W51 region.
(top right) Arecibo 6 cm map of the same region smoothed to the 9.5\arcmin
resolution of the Urumqi data set.  The colorbar applies to both figures,
showing brightness temperature units in K.  The red contours in both figures
show the region observed by Arecibo; flux outside of those boundaries is
extrapolated.
(bottom) Plot of the Arecibo vs the Urumqi surface brightness measurements.
The large red dots show the region within the red contours.
}
{fig:6cmcompare}{0.5}{0}


\subsubsection{Comparison of GBT and Arecibo data}
In order to compare the Green Bank and Arecibo continuum data, we converted
the brightness temperature maps to Janskys assuming a beam FWHM of 50\arcsec
for both surveys and central frequencies of 4.8 and 14.5 GHz for Arecibo and
Green Bank respectively.  Measured beam widths for both telescopes were
$\sim49-54\arcsec$, so the relative error from assuming the same beam size
should be $\lesssim10\%$.
In this section, the target frequencies are referred
to as $S_{5 GHz}$ and $S_{15 GHz}$ for brevity.

The data are well-correlated, with $S_{5 GHz} \sim 1.4 S_{15 GHz}$ ($S_{15 GHz}
\sim 0.7 S_{5 GHz}$; Figure \ref{fig:gbtaocontcompare}), consistent with a
spectral index $\alpha_\nu=-0.3$
slightly steeper than usually observed for optically thin brehmsstrahlung and
consistent with there being some contribution from synchrotron emission.  The
lower-brightness regions have a lower $S_{15 GHz}/S_{5 GHz}$, indicating that
these regions are more affected by synchrotron.  In Figure
\ref{fig:contrrlcompare}a, a great deal of structure in the $S_{15 GHz}/S_{5
GHz}$ ratio is evident in the vicinity of W51 Main: the ratio is higher towards
the continuum peaks, indicating that the peaks have higher free-free optical
depths, or lower relative contributions from synchrotron emission, than their
envelopes.

We additionally compare the radio recombination lines observed simultaneously
with the continuum and \formaldehyde.  Hydrogen RRLs are often extremely
well-correlated with the continuum and are therefore good indicators of the
calibration quality.

In Figure \ref{fig:contrrlcompare}, we show the ratios between the two
frequencies in RRLs and continuum and the line-to-continuum ratios at both
frequencies.  The `line' values are the integrated flux densities over the
range 20 to 100 \kms, which includes all H$\alpha$ emission but no He$\alpha$.

% TODO: re-create h112alpha cube: likely problem with bg subtraction at 20 km/s
% Done!
The ratios between the $x$ and $y$ axis in each plot in Figure
\ref{fig:gbtaocontcompare} are fitted using a total least squares approach with
uniform errors for each data point.   The line-to-continuum ratio is
$L/C(H77\alpha)\sim0.15$ and $L/C(H112\alpha)\sim0.04$; in both cases there is
little evidence for deviation from a linear relationship.

\subsubsection{Comparison of the RRL and continuum data}
\label{sec:rrlvscont}
Radio recombination lines are generally observed to be well-correlated with the
corresponding radio continuum, particularly at low frequencies.  At 5 and 15
GHz, the population level departure coefficients are close to 1, $b_n > 0.95$
\citep{Wilson2009a,Walmsley1990a}.

While radio recombination lines are purely thermal in nature, the large-scale
continuum may include a contribution from synchrotron emission.  The
morphological similarity between the 90 cm and 4 m (meter - i.e., 74 MHz)
images presented by \citet{Brogan2013a} and our 6 and 2 cm data hint that
synchrotron emission could be significant.  However, the high degree of
correlation between the 2 and 6 cm described below suggest that synchrotron
`contamination' is minor at both wavelengths.

Figure \ref{fig:gbtaocontcompare} shows a comparison between the integrated RRL
surface brightness and radio continuum at both 2 and 6 cm\footnote{The H107,
108, 109, and 111$\alpha$ data were affected by missing (corrupted) data in one
segment of the map.  H107 and 108$\alpha$ were also affected by RFI.  We
therefore used the average of the H110 and H112$\alpha$ lines for the 6 cm line
ananylsis.}.  The figure shows
the total least squares best-fit slopes to the data assuming uniform error,
which yield a measurement of the line-to-continuum ratio.

We use the line-to-continuum ratio in both bands to measure the electron
temperature using Equation 14.58 of \citet{Wilson2009a}, which assumes a
plane-parallel, optically-thin emission region with lines formed in local
thermodynamic equilibrium (the $*$ in $T_e^*$ is meant to indicate these three
assumptions are made).  The two lines yield consistent measurements, with mean
$T_e^*\sim7000-8000$ K; these measurements are consistent with smaller-scale
measurements using the VLA with H92$\alpha$ \citep{Mehringer1994a}.  There is
only a little structure in the $T_e^*$ maps, with a hint of higher temperatures
around G49.1-0.4, coincident with the W51C supernova remnant.  Other structures
are most likely due to the limited S/N.

Finally, we fit a single-component Gaussian to each pixel to produce velocity
maps.  These are discussed in Section \ref{sec:kinematics}.

\subsection{Carbon and Helium RRLs}
Helium RRLs were prevalent and reasonably well-correlated with the hydrogen
RRLs, but we did not examine them in detail.  He77$\alpha$ is detected at much
higher signal-to-noise than than He107-112$\alpha$.  There were no clear
detections of C77$\alpha$ or C107-112$\alpha$, though there is a possible
C77$\alpha$ signal at G49.366-0.304 with $v_{lsr} \approx 55$ \kms and a
possible detection toward W51 Main along the wing of the He77$\alpha$ line.
The He77$\alpha$ line detections are associated with regions of high $H_n\alpha$
but not regions of different $T_e^*$.


\Figure{figures/continuum_rrl_comparisongrid.pdf}
{Plots of the 5 GHz and 15 GHz continuum and RRL flux densities against one
another; all units are in Jy.  The dashed lines show the total least squares
best fit line with the slope shown in the legend.  Wherever the density of
points is too high to display, the points have been replaced with a contour
plot showing the density of data points.  The upper-right panel shows a
comparison of the continuum ratio to the RRL ratio.  The dashed line in the
upper-right plot has slope 1, and the dotted line has slope 0.6.}
{fig:gbtaocontcompare}{0.5}{0}

\clearpage
\FigureFour
{figures/ratiomap_cont_c2cmtc6cm}
{figures/ratiomap_cont_h77th112}
{figures/ratiomap_cont_h112lc}
{figures/ratiomap_cont_h77lc}
{Ratio maps of the ionized gas in W51.  
(a) Continuum ratio $S_{15 GHz} / S_{5 GHz}$.  For $\alpha=-0.1$, an optically
thin free-free source, the ratio is 0.9, while for $\alpha=2$, an optically thick source,
the ratio is 9.
(b) The ratio of the H77$\alpha$ peak to the H112$\alpha$ peak.
(c) The line-to-continuum ratio H112$\alpha$ / $S_{5 GHz}$
(d) The line-to-continuum ratio H77$\alpha$ / $S_{15 GHz}$
\todo{A systematic error has appeared in July 2014 that was not previously present.}
}
{fig:contrrlcompare}

\FigureFourPDF
{figures/ratiomap_cont_h112te}
{figures/ratiomap_cont_h77te}
{figures/electron_temperature_77vs111}
{figures/ratiomap_cont_h77te_HeContours}
{(a) The H112$\alpha$ electron temperature map showing $T_e^*$ in K. 
(b) The H77$\alpha$ electron temperature map showing $T_e^*$ in K.
(c) The measured electron temperature in the 6 cm vs the 2 cm band at each spatial
pixel with significant detected RRL emission.  The contours show regions of
increasing pixel density.  The $x$ marks the median and the $+$ marks the mean
over all valid pixels.
(d) Same as (b), but with integrated He77$\alpha$ contours at levels [0.0125,
0.025, 0.05, 0.1, 0.15, 0.2] K \kms overlaid.  The contours on the right side
($\ell<49$) most likely trace noise, since the noise in that region is higher.
}
{fig:tevste}%{0.5}{0}
\clearpage

\subsection{Optical Depths}

The data cubes were converted into ``optical depth'' data cubes by dividing the
integrated \formaldehyde absorption signature by the measured continuum level.
We added a fixed background of 2.73 K to the reduced images to account for the
CMB, which is absent from the images due to the background-subtraction
process.  We define an ``observer's optical depth''
\begin{equation}
    \tau_{obs} = -\ln\left[\frac{T_{mb}}{T_{bg}}\right]
\end{equation}
as opposed to the `true' optical depth, which is modeled in radiative transfer
calculations
\begin{equation}
    \tau = -\ln\left[\frac{T_{mb}-T_{ex}}{T_{bg}-T_{ex}}\right]
\end{equation}
The approximation $\tau_{obs} = \tau$ is valid for $T_{ex} << T_{bg}$, which is
true when an \hii region is the backlight but generally not when the CMB is.
Displaying the data on this scale makes regions of similar gas surface density
appear the same, rather than being enhanced where there are backlights.  The
noise is correspondingly suppressed where backlighting sources are present.
%We will examine the effects
%of using $\tau_{obs}$, i.e. assuming $T_{ex} = 0$, in Section \ref{sec:}.


\formaldehyde absorption is ubiquitous across the map.  In the Arecibo data,
8044 of 17800 spatial pixels have peak optical depths $>5\sigma$, and 14547
have peaks $>3\sigma$, so \formaldehyde absorption is detected at $\sim80\%$ of
the observed positions.

The GBT \formaldehyde \twotwo data have lower peak signal-to-noise because the
continuum background is lower.  Additionally, the \twotwo line is expected to
trace denser gas and therefore be detected along fewer lines of sight.  The
\twotwo line is detected with a peak at $>5\sigma$ in 3497 pixels (20\%) and
$>3\sigma$ in 12254 pixels (69\%).  The high detection rate validates
\formaldehyde as an efficient dense-gas tracer.

There were no detections of \formaldehyde \oneone or \twotwo emission.

\subsection{Ratio Cubes}
In order to compare the \formaldehyde spectra, we first
created optical depth ratio cubes.  The data were masked by selecting all
voxels with signal-to-noise $>2$ in  \emph{both} the \oneone and \twotwo data
cubes.  To filter out spurious signals, of which many are expected given the
3.5 million voxels in the data cubes, we filtered the cube with a
number-of-neighbors filter (i.e., a $3\times3$ kernel with all values of unity
except the center pixel) and removed all pixels with fewer than 7 neighbors
with detections.  After filtering, 35039 voxels were left, or $\sim 1\%$ of the
total (though this fraction is arbitrary, since the number of spectral pixels
is determined by the gridding; i.e. we could have included a larger velocity
range in the output cube).  These are located in 5822 pixels, or 33\% of
the 17718 pixels valid in both data sets.

Histograms of the data are shown in Figure \ref{fig:datahistograms}.  They
illustrate the non-Gaussian nature of the noise.  The ratio distribution, in
particular, appears to follow a power-law distribution with a cutoff at low
ratio.

Note that there is hyperfine structure in the \oneone lines, but the maximum
offset is 1.15 \kms, and that hyperfine line is expected to be 1/6th of the
total.  For the purpose of these bulk comparisons on the large observed scales,
it is safe to ignore the hyperfine structure.

% plot_scripts/tau_histograms.py
\Figure{figures/cube_histograms_tau_and_ratio}
{Histograms of the optical depth data cubes.  In panels (a) and (b),
a Gaussian distribution with width equal to the median absolute deviation
is shown for reference; the noise is not well described by a single Gaussian
distribution.  In panel (c), the distribution of the ratio is shown.}
{fig:datahistograms}{0.4}{0}


\subsection{Converting observed line ratios to volume densities}
\label{sec:models}
The \formaldehyde line ratio can be transformed into a volume
density of hydrogen $n(\hh)$ using large velocity gradient model grids.
The column density of \ortho, the ortho-to-para-ratio of \hh, and the gas
temperature are the three main `nuisance parameters' that can be marginalized
over.

The \ortho column density is degenerate with the velocity gradient in LVG
models.  The \hh column density is degenerate with this gradient \emph{and} the
abundance of \ortho.  Precise measurements of the \formaldehyde abundance are
not generally available, but typical values of $X_{\ortho} = 10^{-10}-10^{-8}$
relative to \hh are generally assumed \citep{Mangum1993a, Ginsburg2011a,
Ginsburg2013a, Ao2013a} and found to be consistent with the observations.
Nonetheless, little is known about local variations in \ortho abundance, except
that it freezes out in cold, dense cores \citep{Young2004a}.

% ~/work/h2co/codes/ratio_to_dens_plots.py
% ~/repos/h2co_modeling/examples/plot_line_brightness.py
% \Figure{figures/tau_ratio_vs_density_varybackground_vary_sigma.pdf}
% {The \emph{observed} optical depth ratio as a function of \hh density.  The
% ratio is shown for $\Xform=10^{-9}$ with two different pairs of background
% temperature.  The first plot shows both lines in absorption against the CMB,
% while the second shows their value for absorption against a bright source with
% a flat spectral index, $\alpha\sim0.1$.
% The gas temperature is assumed to be 20 K.  The gas is assumed to have a
% lognormal density distribution with width $\sigma_s$ \citep[see ][ for
% details]{Ginsburg2013a}, so the X-axis indicates the volume-averaged mean \hh
% density.  A width $\sigma\sim1$ is typical for turbulence with Mach numbers
% $\mathcal{M}\sim1-4$ \citep{Federrath2008a}.  $\sigma=0.01$ is used to
% approximate a delta function.}
% {fig:ratiovsdens}{0.4}{0}

%The curves for converting the optical depth ratio to a density are shown in
%Figure \ref{fig:ratiovsdens}.  
The model grids were generated using RADEX LVG models
\citep[python wrapper \url{https://github.com/keflavich/pyradex/}; original
code][]{van-Der-Tak2007a}.  They assume a velocity gradient of 1 \kms \perpc.
\citet{Ginsburg2011a} and \citet{Ginsburg2013a} discussed the effect of a local
gas density distribution on the molecular excitation, but due to the complexity
involved in accounting for these effects, we ignore them here.


%The curves cover the range of plausible \formaldehyde abundances
%\citep{Mangum1993a}.  While there are multiple densities corresponding to a
%given ratio for $r>8$, we ignore densities $n(\hh)<10^2$ \percc as gas at these
%extremely low densities provide negligible optical depth per particle and
%therefore should not be detected.

%Figure \ref{fig:ratiovsdens} shows three curves that are computed using a
%lognormal volume density distribution with width $\sigma_{\ln \rho}$.  The
%radiative transfer models are computed assuming that each density component
%can be treated separately, e.g. as if they were overlapping slabs of constant
%density along the line of sight.  This approach can only be applied if the
%slabs are optically thin; if radiative coupling effects become significant, the
%feedback between the different density components requires a complete 3D
%radiative transfer approach to model.



%This is all wrong because it assumes T_ex = 0, which is false.
% The upper limit on the observed optical depth ratio indicates a first concern
% with the data: in Figure \ref{fig:datahistograms}, the ratio distribution
% extends smoothly from $\sim15-40$.  These high ratios are excluded by the LVG
% models, but an examination of the data reveals there is nothing uniquely
% problematic about the regions exhibiting high ratios.  While many of the voxels
% are on cloud edges and therefore potentially affect by signal-to-noise issues
% in the \twotwo line, some of the high-ratio regions are contiguous and have
% high signal-to-noise detections in both lines.  The main region affected by
% this systematic issue is the W51 West region, G49.4-0.31.
% 
% The data provide no ready explanations for these high ratios.  In order to
% be consistent with the models, either the \oneone optical depth must be
% too high or the \twotwo optical depth too low.  This could occur if the
% 6 cm continuum is underestimated or the 2 cm continuum is overestimated.
% The latter possibility is very unlikely, as there are no obvious effects that
% would cause this.  The 6 cm continuum could well be underestimated if there
% is a significant diffuse emission component that is completely subtracted off,
% but this is somewhat implausible.  Comparison to the Sino-German 6 cm continuum
% survey \citep{Sun2007a,Sun2011b} reveals excellent agreement with our data at
% all flux levels (see Figure \ref{fig:6cmcompare}); if anything, the 6 cm
% Arecibo data may slightly overestimate the continuum.
% As discussed in Section \ref{sec:gpacompare}, systematic continuum errors at 2
% cm are $\lesssim 0.2$ K, or $<10\%$, and therefore too small to account for the
% difference.
% 
% The data with `unphysical' optical depth ratios are ignored (i.e., set to NaN)
% when converting the ratio cube to a density cube.
% 
% The density distribution across the whole PPV space is shown in Figure
% \ref{fig:ppvdenshist}.  The density distribution
% is reasonably represented by a lognormal with mean $\log(n(\hh))=3.1$ and
% $\sigma=0.5$, but significant bumps at both lower and higher density were
% notable, so we fit a 3-component model as shown in the figure.
% Lognormal distributions are frequently seen in column density plots
% \citep[e.g.]{Kainulainen2011a,Schneider2013a}, and the bump at high density is
% attributed to a power-law tail generated by gravity.  If such a power-law tail
% is present in our PPV density data, it is very subtle, and a set of 3 clouds
% with different mean densities and similar lognormal density distribution widths
% is an equally acceptable statistical representation of the data.
% 
% The few points with $n>10^{4.5}$ all have highly uncertain densities; their
% densities are likely indeed higher than $10^{4.5}$ \percc, but they may have
% nearly any value above that limit, since the \formaldehyde ratio plateaus to 1
% above that density.


% Wrong =(
% \Figure{figures/cube_histograms_density_ppv.pdf}
% {The density distribution of the PPV cube, following the assumptions outlined
% in Figure \ref{fig:ratiovsdens}, particularly $X(\formaldehyde)=10^{-8.5}$.
% The subplots below show the residuals of the fits to the histograms.
% (left)
% The blue curve shows a composite multi-lognormal fit to the distribution,
% with the three individual components shown in red.
% (right) The red curve shows the best-fit single density component.}
% {fig:ppvdenshist}{0.4}{0}

% \subsubsection{Model Uncertainty}
% There is a wide acceptable range of \formaldehyde abundances and \hh density
% distributions to be considered.  Figure \ref{fig:ppvdenshistmulti} shows the
% density distribution derived using two different abundances and two different
% lognormal density distribution widths.  Using a $\sim30\times$ (1.5 dex) lower
% abundance results in a $\sim3\times$ (0.5 dex) higher density, demonstrating
% that our density measurements are fairly robust against even large abundance
% variations.  Changes in the density distribution can have similar effects on
% the inferred mean density; the $\delta$-function distribution shown (which is
% entirely implausible) results in a similar $\sim3\times$ increase in the mean
% density, but a factor of $\sim2$ narrowing in the density distribution.
% 
% \Figure{figures/cube_histograms_density_ppv_multimodel.pdf}
% {Density distributions from the PPV density cube as shown in Figure
% \ref{fig:ppvdenshist}, but for different assumptions about the \formaldehyde
% abundance and lognormal density distribution width.  $\sigma=0$ indicates a
% $\delta$-function distribution.}
% {fig:ppvdenshistmulti}{0.4}{0}
% 
% 
% \subsection{Projections}
% Projections along the velocity axis - peak, sum, or other mathematical
% operations - create simple 2-dimensional images that can be readily compared to
% other data sets and simulations.  Figure \ref{fig:densprojections} shows the
% mean and peak projections along the density data cube.  
% 
% One feature of particularly low mean and peak density appears at $\ell=49.5,
% -0.3<b<-0.2$, while another elongated structure at $48.8<\ell<49, b=-0.3$
% exhibits an overall uniform high density.  The W51 Main region
% ($\ell=49.5$,$b=-0.4$) also shows relatively high densities towards its center,
% but with evidence of a radial decrease in density to its outskirts.


%% projection_figures.py
%\FigureTwoAA{figures/density_mean_projection_withhist.pdf}
%            {figures/density_peak_projection_withhist.pdf}
%{Projections of the derived density along with histograms
%showing the density distribution and best-fit lognormal.
%(top) Projected mean density.
%(bottom) Projected peak density.
%Two main high-density regions are evident: W51 Main at 49.5 -0.4 and G48.9-0.3 at 48.9 -0.3.}
%{fig:densprojections}{1.0}{6in}

% We show a projection of the velocity at peak density in each pixel (Figure
% \ref{fig:velopv}).  This figure clearly highlights the 68 \kms cloud as a
% high-density feature throughout the region \citep{Carpenter1998a}.  However,
% there are significant gaps in the molecular gas, notably a $\sim6 pc$ gap
% (assuming $D=5$ kpc) at $\ell=49.1, b=-0.34$.
% 
% \FigureTwoAA
% {figures/velocity_at_peak_density.pdf}
% {figures/pvdiagram_max_along_latitude.pdf}
% {(top) The velocity at peak density.
%  (bottom) A position-velocity diagram showing the peak density across latitude
%  at each longitude.  There is a density enhancement at the highest velocities
%  across the entire $\sim100$ pc slice; it may be an indication of an impinging 
%  spiral density wave (discussion to be added later XXXX).
% }
% {fig:velopv}{1.0}{6in}

% The position-velocity diagram in Figure \ref{fig:velopv} shows the same general
% feature: the highest velocity gas is the densest.  There appears to be an edge
% of high-density gas tracing the most redshifted velocity at each longitude.
% Given the $\sim100$ pc extent of the cloud and the high-density ridge, it is
% entirely implausible that this structure is caused by any local phenomenon,
% e.g. supernovae or cloud collisions.  Instead, the high-density ridge provides
% evidence for a spiral density wave triggering the high densities.

\subsection{A note on nondetections}
H$_2$ $^{13}$CO was not detected anywhere in the W51 complex in either the
\oneone or \twotwo lines.  The peak signal-to-noise in the \oneone cube was
180, so we report a $3-\sigma$ upper limit on the \formaldehyde/H$_2$ $^{13}$CO
ratio $R>60$, which is consistent with solar values of the $^{12}C$/$^{13}C$
ratio.
% TODO: use integrated_h2co to re-calculate the #

\section{Structure of the Molecular Cloud}
The \formaldehyde observations reveal two essential features of the W51 GMC:
its density structure and its line-of-sight geometry.

\subsection{Molecular Gas and \formaldehyde modeling}
\label{sec:h2co}

Figure \ref{fig:peakoptdepth} shows the most important observed properties of
the \formaldehyde lines.  The figures show the peak observed optical depth
$\tau_{obs} = -\log(T_{MB}/T_{continuum})$ in each line along with the ratio of
the \oneone to the \twotwo optical depth.  They are masked to show significant
pixels determined by:
\begin{enumerate}
    \item Selecting all voxels with $S/N > 2$ in both images or $S/N > 4$ in
        either and with at least 7 (of 26 possible) neighbors also having $S/N > 2$ 
    \item Selecting all voxels with $>=10$ neighbors having $S/N > 2$
    \item Growing (dilating) the included mask region by 1 pixel in all
        directions
    \item Selecting all voxels with $>=5$ included neighbors
    \item When used to mask 2D images, the selection is then collapsed such
        that any pixel containing at least one voxel along the line of sight is
        included
\end{enumerate}
This approach effectively includes all significant pixels and all reliably
detected regions within the data cube, though the number of neighbors used at
each step and the selected growth size are somewhat arbitrary and could be
modified with little effect.

Figure \ref{fig:peakoptdepth} contains two ratio maps.  The first shows the
observed optical depth ratio, while the second shows the `true' optical depth
ratio assuming an excitation temperature for each line, $T_{ex}(\oneone) = 1.0$
K and $T_{ex}(\twotwo) = 1.5$ K.  These excitation temperatures are
representative of those expected for most of the modeled parameter space in
which absorption is expected.  Fitting of individual lines-of-sight confirm
that good fits can be achieved using these assumed temperatures.
%Figure \ref{fig:ratiovsdens} can be applied to Figure \ref{fig:peakoptdepth}d
%with reasonable accuracy, while applying it to Figure \ref{fig:peakoptdepth}c
%would yield incorrect results.

However, there are some clear outliers within the map: the clouds at G48.9-0.3
and G49.4-0.2 both show very low \oneone/\twotwo ratios over a broad area.  As
discussed in Sections \ref{sec:w51b} and \ref{sec:maus}, these two regions have
\hii regions in the foreground of the molecular gas.  The ratios displayed in
Figure \ref{fig:peakoptdepth} are therefore computed with an incorrect
background assumed.


\FigureFourPDF
{figures/peak_observed_opticaldepth_11}
{figures/peak_observed_opticaldepth_22}
{figures/peak_observed_opticaldepth_ratio}
{figures/peak_observed_opticaldepth_ratio_tex}
{Plots of the peak \emph{observed} optical depth $\tau_{obs} =
-\log(T_{MB}/T_{continuum})$ in the (a) \oneone and (b) \twotwo lines and (c)
their ratio, \oneone / \twotwo.  Figure (d) shows the `true' optical depth ratio
assuming $T_{ex}(\oneone) = 1.0$ K and $T_{ex}(\twotwo) = 1.5$ K; these are
reasonable and representative excitation temperatures but they are not fits to
the data.
The data are masked with a filter described in Section \ref{sec:h2co} and cover
the range $75 > V_{LSR} > 40$ \kms; see Figures \ref{fig:lowerpeakoptdepth} and
\ref{fig:upperpeakoptdepth} for individual velocity components.  In general,
lower (redder) ratios in figures (c) and (d) indicate higher densities, however
in the filament at 49.0-0.3, the low ratio is due to the geometry in which
$T_{continuum}$ is in the \emph{foreground} of the molecular gas.}
{fig:peakoptdepth}


% plot_scripts/opticaldepth_plots.py
\FigureFourPDF
{figures/lowerpeak_observed_opticaldepth_11}
{figures/lowerpeak_observed_opticaldepth_22}
{figures/lowerpeak_observed_opticaldepth_ratio}
{figures/lowerpeak_observed_opticaldepth_ratio_tex}
{Same as Figure \ref{fig:peakoptdepth}, but limited to $62 > V_{LSR} > 40$ \kms.}
{fig:lowerpeakoptdepth}

% plot_scripts/opticaldepth_plots.py
\FigureFourPDF
{figures/upperpeak_observed_opticaldepth_11}
{figures/upperpeak_observed_opticaldepth_22}
{figures/upperpeak_observed_opticaldepth_ratio}
{figures/upperpeak_observed_opticaldepth_ratio_tex}
{Same as Figure \ref{fig:peakoptdepth}, but limited to $75 > V_{LSR} > 62$ \kms.}
{fig:upperpeakoptdepth}

\subsection{Density Maps}
\label{sec:densmaps}
We computed the density using the $\chi^2$ minimization technique from
\citet{Ginsburg2011a}.  We measure $\chi^2$ over the full 4D parameter space (density, column density, temperature,
and ortho-to-para ratio)
\begin{equation}
    \label{eqn:chi2}
    \chi^2 =  \left( \frac{T_B(\oneone)-T_{model}(\oneone)}{\sigma(T_B,\oneone)}\right)^2 +
              \left( \frac{T_B(\twotwo)-T_{model}(\twotwo)}{\sigma(T_B,\twotwo)}\right)^2
\end{equation}
We have therefore not enforced any constraints on the column density,
temperature, or ortho-to-para ratio when fitting.
The best-fit value of each of these parameters is taken to be the mean of those
parameters over the range $\chi^2 - \chi^2_{min} < 1$.

The temperatures returned from the fitting process are, as expected, purely
noise: the \formaldehyde \oneone/\twotwo ratio provides virtually no constraint
on the gas temperature and therefore leaving it as a free parameter has no
effect on the fitted density.  Similarly, the ortho-to-para ratio of \hh is
unconstrained in our data.  In principle, the \hh OPR has some effect on
\formaldehyde excitation, but in the regime we have modeled and obseved, no
effect is apparent.


The \ortho-column-weighted volume-density along each line of sight is shown in
Figures \ref{fig:wtdmeandens} and \ref{fig:wtdmeandensvrange}.  The former
shows the weighted density over all voxels and the latter shows the weighted
density over the two velocity ranges previously discussed.  These projections
include no information about the errors in the individual fits, which are
available from Equation \ref{eqn:chi2}, but by weighting by column density, we
have effectively selected the highest signal-to-noise points; the statistical
errors are therefore negligible relative to the systematic in these maps.



The overall picture is of a central proto-cluster region with most of the gas
mass at a density $n\sim10^{5.5}$ \percc within a diameter of $\sim3$ pc, surrounded
by a rich cloud in which most of the mass is at a density $\sim10^4$ \percc out
to a diameter $d\sim14$ pc.


% parplots_chi2gridfit.py
\Figure
{figures/H2CO_ParameterFitPlot_weighted_mean_meanfit_density_mean_linear.pdf}
{Map of the column-weighted volume density along the line of sight averaged
over all velocities.}
{fig:wtdmeandens}{0.5}{0}

% parplots_chi2gridfit.py
\FigureTwoAA
{figures/H2CO_ParameterFitPlot_weighted_mean_meanfit_density_v40.0to62.0_mean_linear.pdf}
{figures/H2CO_ParameterFitPlot_weighted_mean_meanfit_density_v62.0to75.0_mean_linear.pdf}
{Map of the column-weighted volume density along the line of sight, split into 
(a) the 40-62 \kms component and (b) the 62 to 75 \kms component.}
{fig:wtdmeandensvrange}{1}{6.5in}

\subsection{Model fitting and geometry}
Both \formaldehyde lines are seen only in absorption.  However, in some cases
the absorption is against a continuum background, while in others the
absorption may be only against the CMB.

We have fit the \formaldehyde lines constrained by the LVG models (Section
\ref{sec:models}) to spectra averaged over regions with coherent molecular
absorption signatures.  We compared the $\chi^2$ values for fits with the
observed continuum as the background to those with the background fixed to
$T_{BG} = T_{CMB}$.  We then selected the better of the two fits as
representative of the physical conditions.

It is possible that there are multiple continuum emitters along the line of
sight in many cases, with the absorbing molecular gas somewhere in the middle.
While this possibility adds uncertainty to the measurements, there are some
cases in which the dominant continuum can unambiguously be assigned a
foreground or background position.

%\FigureSVG{figures/geometry_diagram.pdf_tex}
%{A sketched diagram of the W51 region}
%{fig:geosketch}{6cm}
\Figure{figures/geometry_diagram.png}
{A sketched diagram of the W51 region as viewed from the Galactic north pole,
with the observer looking up the page from the bottom (i.e., W51C is the 
front-most labeled object along our line-of-sight).  There are a few
significant differences between this and Figure 29 of \citet{Kang2010a},
particularly the relative geometry of the cloud and the \hii regions in W51 B.
We also show a good deal more detail, revealing that there are \hii regions on
both front and back of many clouds.
\todo{This figure needs to be kept up to date with text changes.  Ideally, I'd
like to build an interactive version of this figure, but that may not happen in
finite time.}}
{fig:geosketch}{0.5}{0}

\Figure
{figures/W51_wisecolor_labeled_detail.pdf}
{Labeled figures of the W51A and W51B regions, highlighting \hii regions and
infrared dark clouds.   The colors are described in Figure \ref{fig:w51large}.
These labels can be compared to Figure \ref{fig:geosketch} to associated
labeled regions in the plane of the sky with their counterparts in the face-on
view of our Galaxy.}
{fig:labeledzoom}{0.5}{0}


\subsection{Kinematic Maps}
\label{sec:kinematics}
Maps showing the overall kinematics of the region are shown in Figures
\ref{fig:kinematics} and \ref{fig:h2cokinematics}.  Figure \ref{fig:kinematics}
shows the velocity at peak absorption of the \formaldehyde \oneone line and the
fitted radio recombination line centroid velocity.  Figure
\ref{fig:h2cokinematics} shows the best simultaneous fit to the \formaldehyde
\oneone and \twotwo absorption features over two different velocity ranges.
The \oneone absorption velocity in Figure \ref{fig:kinematics}a approximately
shows the velocity of the front-most molecular clouds along the line of sight
at each position.

% plot_scripts/h2co_velo_images.py
% plot_scripts/rrl_images.py
\FigureTwoAA
{figures/H2CO11_central_velocity}
{figures/H110a_central_velocity}
{(left) Velocity of the peak \formaldehyde \oneone signal (deepest absorption) at
1 \kms resolution
(right) Velocity of the peak H110$\alpha$ emission as derived from Gaussian fits
to each spectrum.}
{fig:kinematics}{1}{6.5in}

% parplots.py
\FigureTwoAA
{figures/W51_H2CO_2parfittry10_velocity1}
{figures/W51_H2CO_2parfittry10_velocity2}
{Maps of the fitted \formaldehyde velocity components over the range $40 <
v_{LSR} < 66$ \kms (left) and $66 < v_{LSR} < 75$ \kms (right).  The regions
that appear noisy have ambiguous multi-component decompositions.  \todo{This is
being refit}}
{fig:h2cokinematics}{1}{6.5in}

\subsection{Geometry of Individual Regions}
\subsubsection{The W51 B Filament}
\label{sec:w51b}
The W51 B filament, labeled as the 66 \kms cloud in Figure \ref{fig:geosketch},
exhibits bright CO emission but has relatively weak \formaldehyde absorption.
The absorption models are inconsistent with the molecular gas being in front of
the continuum emission, so Figure \ref{fig:geosketch} shows the continuum sources
in front of the cloud at lower $\ell$.  Figure \ref{fig:h2cofrontbackmodel}
shows an example model fit with the continuum assumed to be in front and in
back, illustrating that the best-fit model parameters with continuum in the
back do not reproduce the data.
The relative positioning of the molecular gas behind the \hii regions suggests
that the molecular gas is also behind the W51 C supernova remnant.

% pyspeckit_individual_fits
% paperfigure_filament_demonstrate_frontback(fixedTO=True):
\Figure{figures/spectralfits_70kmscloud_aperture_ap6_modelcomparison_withresiduals.pdf}
{An example of the difference in models between a continuum source (red) and
the CMB (green) as the background.  The top plot shows the \oneone line and the
bottom shows the \twotwo line both with the continuum level set to zero in the
plot but treated as a frozen parameter in the fit.  The residuals are shown
offset above the spectra, with the dashed line indicating the zero-residual
level.  The grey shaded regions show the 1-$\sigma$ error bars on each pixel.
The model with the CMB as the only
background is able to reproduce the absorption line, while the model with the
\hii region in the background cannot account for the depth of the \twotwo line.
The reduced $\chi^2/n$ for the models are 14.1 (red) and 2.8 (green), evaluated
only over the pixels where the model is greater than the local RMS.}
{fig:h2cofrontbackmodel}{0.5}{0}

% The mass-weighted mean density ranges from $\sim$1.5\ee{4} to 5.6\ee{4} \percc,
% with the lowest mean density corresponding to the highest column density toward
% the center.


\subsubsection{The edge of W51 C}
W51 C is a supernova remnant that spatially overlaps with the W51 B star
forming region.  \citet{Brogan2013a} argue that the supernova remnant must be
in front of the \hii region G49.20-0.35 because the \hii-region has not absorbed
all of the 4m (74 MHz) nonthermal emission.  The G49.1-0.4, G49.0-0.3, and G48.9-0.3
regions, however, show 4m absorption signatures and may be in the foreground.
There are clumps aligned along the 68 \kms filamentary cloud with very high CO
and \hi velocities \citep{Koo1997b,Koo1997c,Brogan2013a}, indicating that the
SNR is interacting with the molecular gas.

The clumps at G49.1-0.3, $\sim68$ \kms are either lower density ($n<1.5\ee{4}$
\percc) and in the background of the \hii region or high density ($n>1.5\ee{5}$
\percc), low-column density and in the foreground.  The $62$ \kms clumps have
densities a few times higher, $n\sim4\ee{4}$ \percc, and are clearly in the
foreground of the continuum emission because their absorption depths are
$\sim2.5$ K, which cannot occur for absorption against the CMB.  Figure
\ref{fig:contbetween63kms} shows a model spectrum fitted assuming the continuum
lies between the two molecular velocity components.  The relative strength of
the \thirteenco and the \formaldehyde also suggests that the 68 \kms component
is behind the continuum.

We are seeing molecular gas both in front of and behind the supernova.  This
geometry can be readily confirmed by looking for molecular absorption at much
lower frequencies where the SN synchrotron emission dominates over the \hii
region free-free emission, i.e. the 335 and 71 MHz \para lines.

\Figure{figures/spectralfits_63kmscloud_aperture_ap3_both_legend.pdf}
{The spectrum extracted from G49.119-0.277 in a 55\arcsec radius aperture,
showing a model in which the continuum is \emph{behind} the 63 \kms component
but in front of the 68 \kms component.  The legend gives the fit parameters
along with $1-\sigma$ error bars.  The parameters with no errors indicated
(OPR, $T$, $T_{BG}$) are assumed or independently measured values.}
{fig:contbetween63kms}{0.5}{0}

\subsubsection{G49.20-0.35 and G49.1-0.4}
\citet{Tian2013a} focus on the \hii regions G49.20-0.35 and G49.10-0.40 (called
G49.10-0.38 in their work) to determine the relative geometry
of the W51 C SNR and the W51 B \hii/star-forming region.  They observe that the
high-velocity \hi is not detected toward either of these sources, indicating
that the \hii regions must be behind the high-velocity \hi features.

We detect \formaldehyde \oneone at $\sim58$ and $\sim63$ \kms toward
G49.10-0.40, with line ratios that are consistent with the \hii region being
behind the molecular cloud complex.  It also has an extreme RRL velocity,
$v_{110\alpha} \approx 72\kms$, the most redshifted seen in the entire W51 region
(see Figures \ref{fig:kinematics} and \ref{fig:hiirrlspec}).

G49.20-0.35 is also clearly behind the molecular cloud, as evidenced both by
\formaldehyde absorption depth and the IRDC absorption in the foreground.
It has an RRL velocity $v_{110\alpha} \approx 70$ \kms.

Because both \hii regions are extremely redshifted, they are most likely
associated with the W51 B cloud complex, contrary to the interpretation by
\citet{Tian2013a} in which they are unrelated background clouds.  The Galactic
rotation curve doesn't allow for velocities red of $\sim60$ \kms, and almost
none of the molecular gas exceeds $\sim70$ \kms even on the wings.  The \hii
regions are therefore probably shooting out the back side of the molecular
cloud, perhaps accelerating ionized gas from the $\sim 66$ \kms component
further to the red.

\FigureTwo
{figures/spectralfits_70kmscloudLeft_wideplot_RRL_aperture_G49.20-0.35.pdf}
{figures/spectralfits_70kmscloud_wideplot_RRL_aperture_G49.1-0.4.pdf}
{Fitted H110$\alpha$ (red) and H77$\alpha$ (black) spectra extracted from
55\arcsec apertures centered on G49.20-0.35 (left) and G49.1-0.4 (right).
The best-fit Gaussian parameters are shown in the legends, with the lower
legend corresponding to H77$\alpha$.}
{fig:hiirrlspec}{1}

\subsubsection{The 66 \kms IRDC}
Between W51 A and W51 B, there is a component of the 68 \kms cloud that is
filamentary and in the foreground of all of the free-free emission.
This cloud component is evident as an IRDC in the Spitzer GLIMPSE images from
$\ell=49.393$, $b=-0.357$ to $\ell=49.207$, $b=-0.338$.

The \hii region G49.20-0.35 is clearly behind the IRDC, though there are strong
morphological hints that it is interacting with and truncated by the cloud.

% analysis_scripts/pvfigure_h2co13co.py
\FigureTwo
{figures/filament_extraction_region_on_HiGal.pdf}
{figures/cyan_wide_PV_h2co11_22_13co.pdf}
{(left) A column density map fitted from the Herschel Hi-Gal data with two filament extraction regions superposed
in cyan and red-blue
(right) A position-velocity slice of the 68 \kms cloud, shown in cyan in the
left figure, which includes an infrared dark cloud and the interaction region
with the W51C supernova remnant.
(top) \formaldehyde \oneone observed optical depth
(middle) \formaldehyde \twotwo observed optical depth
(bottom) \thirteenco 1-0 emission from the GRS with \formaldehyde \oneone
contours superposed.  The weakness of the \formaldehyde absorption on the right
half of the cloud corroborates the geometry inferred from comparison of the \oneone
and \twotwo lines in Figure \ref{fig:h2cofrontbackmodel}.}
{fig:filament_pvslice}{1}%{0}

\subsubsection{G49.27-0.34}
% digging_in_to_G49.27-0.34.py
The \uchii region G49.27-0.34, which was considered a candidate extended green
object (EGO) and subsequently rejected for lack of \hh emission
\citep{De-Buizer2010a,Lee2013a}, exhibits a second velocity component at $\sim68
\kms$, slightly but clearly redshifted of the rest of the IRDC.  It contains a
gas mass $\sim2\ee{3} \msun$ based on the BGPS flux and using the assumptions
outlined in \citet{Aguirre2011a}, suggesting that the high velocity could be
due to infall or virialized gas within a deep potential.  The virialized
velocity width, given the radius and mass from the BGPS data, is
$\sigma_{vir}=8.8 \kms$, while the measured \formaldehyde linewidth is
$FWHM(\formaldehyde) = 7.2 \kms$, wider than in any other part of the cloud
except W51 Main.

Both radio continuum and RRLs are detected toward this source.  The H77$\alpha$
RRL velocity is $\sim58$ \kms, significantly blueshifted from the molecular
gas.  The \formaldehyde lines do not independently distinguish between the
continuum source being in the front or back of the cloud, but the mean density
from the BGPS mass and radius $n\sim2.5\ee{4}$ \percc is within a factor of 2
of the \formaldehyde-derived density, $n\sim1.4\ee{4}$ \percc, if the continuum
source is behind the gas, while the \formaldehyde-derived density is too low,
$n\sim2\ee{3}$ \percc if the continuum source is in front.

The implied geometry therefore has the \hii region behind the molecular gas,
plowing toward it at a velocity difference $\Delta v \sim 10$ \kms.  Such a
high velocity difference may indicate that the \hii region is confined by the
molecular gas and on a plunging orbit into the cloud.
% Notes to self:
% Considered the alternative: what if the cloud was actually much hotter and
% therefore lower mass?  Then, the HII region could be in front.  That would
% also explain the CO self-absorption.  However, the temperatures required are
% ~50-100 K, but the CO temperature is ~15K
% Could do an SED fit to check; that is not worth the effort at present

\subsubsection{G49.4-0.3f, aka G49.34-0.34, aka IRAS 19209+1418}
% 70kmscloudLeft ap1 -> G49.34-0.34
The \hii region centered at 49.34-0.34 was identified by \citet{Mehringer1994a}
as part of the G49.4-0.3 complex.  There are 3 distinct \formaldehyde line
components at 51, 63.70, and 68.47 \kms.  The 51 \kms component is behind the \hii
region; the \thirteenco line is detected at comparable brightness at 51 \kms
and 63 \kms, while the \formaldehyde \oneone line is $\sim10\times$ deeper at
63 \kms.  The RRLs associated with this source are at $v_{LSR}=58 \pm 1$ \kms.

The \formaldehyde lines are moderately well-fit by the two-velocity-component
model, but there is a relative excess of \twotwo absorption at 66 \kms.  The
extra absorption may indicate that there is a high-density, low-column
component at this velocity.

The 8 \um GLIMPSE image shows that the 68 \kms IRDC crosses in front of this
source.  Herschel Hi-Gal 70\um images reveal a ring structure that is hinted at
in the 8 \um image.  There is no evidence for interaction between the ring
feature and the IRDC.  This intriguing feature will likely be difficult to
study in detail because the dusty, molecular gas feature lies in front of it.

\subsection{G49.4-0.3}
\label{sec:maus}
The collection of \hii regions around G49.4-0.3 vaguely resembles a cartoon
mouse.  As noted in \citet{Carpenter1998a}, the molecular gas in this region is
separated into two distinct components, one at 51 \kms and the other at 64
\kms.  The 64 \kms component is in the foreground, while the 51 \kms is in the
background of most of the \hii regions.  

Both cloud components are in the foreground of the central \hii regions at
G49.36-0.31, the `eyes' of the mouse.  The density of the 51 \kms component is
an order of magnitude higher than that in the 64 \kms component in this region,
suggesting that the gas is being compressed by the \hii region.
The clean separation between the 64 and 51 \kms cloud components suggests that
they are not interacting at this location.  

Based on the absorption line depths, the G49.38-0.30, IRAS 19207+1422, and
G49.37-0.30 \hii regions are behind the 51 \kms cloud.  The 8 \um absorption
features are associated with the 64 \kms cloud and are in front of all of the
\hii regions.

The 8 \um morphology of G49.42-0.31 is bubble-like, so it is plausible that the
\hii region is neither in front nor behind the 51 \kms cloud but embedded
within it, blowing a hole in the cloud.

\subsection{Infrared Dark Cloud G49.47-0.27}
The cloud to the north of W51 Main/IRS2 appears as a dark feature in Spitzer
GLIMPSE 8 \um maps.  It is detected in \formaldehyde from 54 to 64
\kms.  Throughout, it has a high \oneone/\twotwo ratio, $\gtrsim7$ in
most voxels, indicating a low density $n\lesssim10^3$ \percc. 

%The \oneone
%optical depth is high, up to $\sim1/3$, so if there is any clumping, the gas
%may be optically thick, which would imply that our density measurement is an
%overestimate.

Centered at 60.6 \kms, the region has a line FWHM 5 - 7 \kms, indicating that
it is quite turbulent, with 3D Mach number in the range $10 < \mathcal{M} < 20$
for an assumed $10 < T < 20$ K.  At its centroid velocity, it is connected
to the W51 Main cloud.

There is a previously unreported bubble HII region in the north part of this
cloud, which we designate G49.47-0.26, with radius $\sim70\arcsec$ (1.7 pc).
The \hii has RRL velocities $v_{lsr}\approx50$\kms.  Because it is not detected
in Brackett $\gamma$ emission \citep[from the UWISH2
survey:][]{Froebrich2011a}, it is most likely behind the cloud.

The \citet{Kang2009a} Spitzer survey of YSOs in the region indicates that there
are no YSOs within the boundaries of this cloud; it is very likely
non-star-forming at present.

Because the cloud is continuous with the W51 Main region in velocity and is
infrared-dark, it is most likely at the same distance as W51 Main and
associated indirectly with the massive cluster forming region.

\Figure
{figures/purple_wide_PV_h2co11_22_13co.pdf}
{Position-velocity diagrams of filamentary structures to the north and south of W51 main.
See Figure \ref{fig:filament_pvslice} for the extracted region.}
{fig:northsouthpv}{0.5}{0}

%\subsection{HII region G49.37-0.30, AKA W51 West}
%W51 West is a busy and luminous \hii region; it is the second-brightest radio
%and millimeter continuum source in our survey after the W51 Main/IRS2 region.
%As noted in \citet{Carpenter1998a}, its velocity ($v_{LSR}\sim 50\kms$ XXX) 
%may indicate that the cloud is unassociated with the W51 Main GMC.

% \subsubsection{High density gas in G49.5-0.4?}
% \label{sec:g495}
% \todo{This section is too detailed and irrelevant unless I can draw conclusions.
% These paragraphs are ``working notes'' and will be removed.}
% The region G49.47-0.42 also appears to have a very low ratio.  In this region,
% there are two deep lines of \formaldehyde \oneone, both indicating absorption
% against an \hii region.  Curiously, toward this region, the deeper \twotwo line
% corresponds to the shallower \oneone line, so the ratio
% is nonphysical.  However, when comparing the optical depth of the two line
% components directly (Figure \ref{fig:lowerpeakoptdepth}), the deeper \twotwo
% component remains peculiar: the simple RADEX LVG models cannot accomodate this
% deeper \twotwo line with any combination of density, temperature, column
% density, and ortho-to-para ratio.  There are a few possible explanations for 
% this difficulty:
% \begin{enumerate}
%     \item The \twotwo line is overestimated because of calibration issues
%     \item There is a significant emission component affecting the \oneone
%         line, making it appear weaker (shallower) than it really is
%     \item Multiple density and/or velocity components are present
%     \item Multiple continuum sources exist along the line of sight, one
%         optically thin dominating the 6 cm emission and the other optically
%         thick, dominating the 2 cm emission.
% \end{enumerate}
% Option (1) is implausible, since the corresponding line at high velocity (68
% \kms) shows no peculiarities.  Option (2) is possible, though extant VLA
% observations seem to rule it out, and there is a very narrow range of parameter
% space in which \oneone emission would occur simultaneous with \twotwo
% absorption, especially given the bright ($\sim22$ K) background at 6 cm.  After
% testing many potential combinations of parameters, we have ruled out option
% (3).  Option (4) is seemingly possible, although the required spectral index
% of the 6 cm source is $\alpha\sim -2$, which is a much steeper inverse spectral
% index than is ever observed elsewhere.  Option (4) points to the possibility
% that there is some systematic error in the continuum of perhaps the 6 cm data,
% there is no hint of such an issue in the RRL comparison in Section
% \ref{sec:rrlvscont}.


\subsection{W51 Main \& W51 IRS 2}
The W51 Main and IRS 2 spectra show that both have ionized gas components at
$v_{LSR}\sim 55$ \kms.  This velocity approximately coincides with the peak of
the \thirteenco emission.

The \formaldehyde \oneone spectra are deepest at $\sim68$ \kms, while the
\twotwo have depths approximately equal between the $\sim58$ \kms and $\sim 68
\kms$ components.  The 55-60 \kms components are too deep to be entirely behind
the \hii regions.  This indicates that the 55 \kms ionized gas must be embedded
within the molecular cloud, with molecular gas on \emph{both} sides of the
ionized gas along the line of sight.

Because these are well-studied regions, the low spatial resolution
\formaldehyde spectra we present here add little new information about the gas
kinematics.  However, all of the velocity components observed in the W51 region
are apparently kinematically connected to the W51 clusters.

\subsection{The 40 \kms clouds}
There are clouds observed at 40 \kms that show only weak \formaldehyde
absorption spread across nearly the entire region.  These molecular clouds are
behind nearly all of the \hii regions in the W51 complex.  There are additional
40 \kms clouds clearly seen in \hi absorption \citep{Stil2006a} that are not
associated with these molecular clouds, but instead represent a foreground
population of neutral atomic medium clouds.



% % plot_scripts/parplots.py
% \FigureThreePDF
% {figures/W51_H2CO_logdensity_textbg_max_ratio_sigma0.5}
% {figures/W51_H2CO_logdensity_textbg_mid_ratio_sigma0.5}
% {figures/W51_H2CO_logdensity_textbg_min_ratio_sigma0.5}
% {Density maps of the W51 cloud complex.  The density maps are created by
% mapping the (a) maximum (b) median (c) minimum observed optical depth ratio to
% density using the RADEX models with the measured background temperature.  The
% models used assume $\Xform = 10^{-9}$ and $\sigma_s = 0.5$; a range of model
% choices is valid but they all yield systematic shifts.  These figures show that
% a moderate-density component ($n(\hh)\sim3000$ \percc) is present in at least
% one spectral bin at each position, while some regions have peak densities
% exceeding $n>10^6$ \percc.
% We accounted for the relative line-of-sight location of the continuum sources
% by setting the background to 2.73 K within the G48.9-0.3 and G49.4-0.2 regions.
% We did not attempt to correct the continuum in locations where the continuum
% source is between two clouds.
% The apparently high peak density of the G49.5-0.4 region is discussed in greater
% detail in \ref{sec:g495}.
% }
% {fig:densitymaps}

% Describe fitting models to absorption lines (NOT to tau)
% 
% A local cloud at $v_{lsr}\sim5 \kms$ is detected in \formaldehyde \oneone
% across most of the cloud and not detected at \twotwo, with
% $\tau_{\oneone}/\tau_{\twotwo} \gtrsim 3$, implying a very low column
% $N_{\formaldehyde}\sim10^{11.5}$ or $N_{\hh} \sim 10^{20.5}$.  The cloud is
% seen in \thirteenco as a very weak, diffuse feature, and in HI absorption as a
% narrow (self)-absorption feature.

%% Pixel fitting doesn't look as good as the above approaches.  Let's stick with
%% them.
% \subsection{Pixel Fitting}
% We fit the data with \formaldehyde models on a pixel-by-pixel basis.  Nearly
% all pixels with significant detections include two blended velocity components.
% Two-component fits are never particularly stable, so it was necessary to
% restrict the parameters being fitted.
% 
% The model fitting provides a complementary view to the optical depth ratio maps
% shown in Figure \ref{fig:peakoptdepth}.  Regions of lower signal to noise
% appear scattered in Figure \ref{fig:w51h2cofits}, but each pixel contains
% information about the velocity, velocity width, column (\perkms \perpc), and
% density.
% 
% 
% \FigureTwo{figures/W51_H2CO_2parfittry10_v1_densityvelocity.pdf}
%           {figures/W51_H2CO_2parfittry10_v2_densityvelocity.pdf}
% {Density and velocity fits to the \formaldehyde 
% data cubes.  The left figure shows velocity components restricted
% to have central $V_{LSR} < 65$ \kms, while the right figure shows
% velocity components with $V_{LSR} > 65$ \kms.  
% %The yellow regions in the top panel correspond to \oneone
% %detections and \twotwo nondetections, indicating upper limits $n<10^{3.8}$
% %(68\% confidence) or $n<10^{4.3}$ (99.7\% confidence).
% }
% {fig:w51h2cofits}{1}

\section{A formation scenario for the W51 region}
The W51 cloud complex has been discussed as both a collection of unrelated
clouds and a tight complex of interacting clouds \citep{Carpenter1998a,
Kang2010a}.  The \formaldehyde data presented here support the idea that the 68
\kms cloud is a coherent entity and that it is interacting with other clouds
associated with W51 A.  The presence of W51 Main at the interaction point
between multiple clouds hints that its great mass and star-forming potential
was triggered by this cloud-cloud collision.  It is likely that both observed
clouds were pre-existing features in a larger, possibly atomic, cloud that
underwent stretching and squeezing upon interaction with a spiral arm.

%Alternatively, the W51 complex could trace a `spur' from a spiral arm.  In
%other galaxies, e.g. M51, spurs are frequently seen to be richly star-forming,
%sporting large \hii regions.
%Spurs are typically compact ($\lesssim 50$ pc)
%along their short axis, which would be broadly 


The line-of-sight length of the W51 complex is still uncertain, despite our
constraints on the relative geometry of different regions.  The best prospect
for resolving the line-of-sight structure of the region is via precise
constraints on distances to the individual regions.  Spectrophotometric surveys
of the individual stellar sub-clusters may be able to provide this and should
be undertaken.  Maser parallax observations of different zones may also provide
differential distance estimates.

\section{Discussion}
The W51 cloud complex includes a full range of star forming conditions.  In the
west, W51 B, there is an older generation of stars including at least one supernova
remnant.  In the east, there is a pair of forming, still-embedded massive clusters.
We have described the geometry of these regions and features of the cloud structures,
now we will speculate on the broader implications of these observations.

The gas in the W51 B region, while clearly affected by the expanding W51 C
supernova, is less dense than most of the gas in the W51 A region.  The
supernova feedback is, if anything, destructive; a `collect-and-collapse'
scenario does not fit the observed gas structure.  Stellar feedback in the W51
region is therefore destructive, though so far not severely so.

The proximity of the 68 \kms filamentary `high velocity stream' and the W51
Main protocluster and their relative line-of-sight positions have been
presented as evidence for a cloud-cloud collision \citep{Kang2010a}.
Examination of the \formaldehyde line ratios has shown that the protocluster is
embedded in the $\sim55$ \kms molecular cloud.  

%\todo{Is the W51 filament like Nessie?  Too coherent over large scales,
%flattened in the midplane?}

\subsection{Gas density and its relation to star formation}
The lowest density readily observed in our survey of W51 is $n\sim10^4$ \percc.
Lower densities can be detected with \formaldehyde observations but require
greater depth, especially in the \twotwo line.

In \citet{Ginsburg2013b}, we measured the \formaldehyde line ratio to high
precision in a low-density, turbulent, non-star-forming GMC.  While the mean
density of this cloud was of order $n\sim10^2$ \percc, the mass-weighted
density as probed by \formaldehyde was $n\approx5\ee{3}$ \percc.  The density
measurements in the W51 cloud exceed this value for the majority of the
detections significant in both \oneone and \twotwo, indicating that there is a
much higher fraction of high-density gas in this star-forming cloud than in the
quiescent GMC in front of W49.

\subsubsection{Dense Gas Mass Fractions}
Because the \formaldehyde densitometer yields a mass-weighted measurement of
the gas volume density, it is difficult to connect directly to the total gas
mass, which is the quantity of interest when determining bulk properties like
star forming efficiency.  However, because the \formaldehyde and CO trace the
same gas, the \formaldehyde-derived density can be applied to the total mass
measured by CO.  We assume that each \thirteenco PPV `voxel' has a mass
proportional to its integrated intensity and a density given by the $n(\hh)$
delivered from the \formaldehyde densitometer.

The `dense gas mass fraction' (DGMF) is an oft-quoted measurement used to argue
about the speed of the star formation process, the existence of density
threshold, and turbulent properties of the ISM \citep[e.g. Fig. 5
of][]{Krumholz2007a,Battisti2014a,Kainulainen2013a,Juneau2009a,Muraoka2009a,Hopkins2013e}.
However, these fractions are most often quoted as mass of gas at a \emph{single}
density divided by the total mass.  We present an improvement on these measurements,
showing the continuous distribution of the dense gas mass fraction.

Figure \ref{fig:dgmf} shows the result of using our \formaldehyde PPV density
cubes to measure the DGMF from \thirteenco.  We use a range of density
thresholds from $\sim10^3$ to $\sim10^4$ \percc.  At each density, we identify
all pixels in the \thirteenco PPV cube above this threshold and sum those.  We
then divide by the total integrated \thirteenco brightness to get the mass
fraction. 

Figure \ref{fig:dgmf_regions} shows the same results, but for two individual
regions: the W51 Main proto-cluster and the W51 B region.  Within about 10 pc
of W51 Main, around half of the mass is at density $n>10^4$ \percc.  By
contrast, the rest of the molecular cloud shows a consistent fraction
$f\sim10\%$ with $n>10^4$ \percc.


\Figure{figures/FractionOfMassAboveDensity_All}
{The `dense gas mass fraction' as a function of volume density $n(\hh)$ \percc.
The $y$-axis shows the sum of \thirteenco pixels from the GRS cube with
\formaldehyde-derived density above the value shown on the $x$-axis divided by
the total.  Both values are computed over the velocity range $40 \kms < v_{LSR}
< 75 \kms$.  The blue shaded region shows the extent of plausible model fits at
each density: effectively, this is the $\sim1-\sigma$ error region.  The
vertical line at $n=10^4$ \percc indicates the approximate completeness limit.
The horizontal line shows the fraction of \thirteenco flux detected at
$>2\sigma$ in both the \oneone and \twotwo lines: it represents the upper limit
of what could have been detected if, e.g., all \formaldehyde detections were
toward regions with $n>10^4$ \percc.  The failure to converge to a fraction
$f\rightarrow f_{max}$ indicates that there are some real detections of
low-density gas.}
{fig:dgmf}{0.5}{0}

\FigureTwo
{figures/FractionOfMassAboveDensity_W51Main}
{figures/FractionOfMassAboveDensity_W51B}
{Same as Figure \ref{fig:dgmf}, but for two individual regions: W51 Main
(left), the region $49.4\arcdeg < \ell < 49.6\arcdeg$, $-0.5\arcdeg < b <
-0.3\arcdeg$, and W51 B (right) with $\ell < 49.4\arcdeg$.  Note
that the $y$ axes have different ranges.}
{fig:dgmf_regions}{1}

\FigureTwoAA
{figures/DGMF_5e4_Contours_on_13CO.pdf}
{figures/DGMF_1e4_Contours_on_13CO.pdf}
{Contours of the dense gas mass fraction using two different thresholds
overlaid on the integrated \thirteenco map.  The regions with fraction $f>0.5$
should be rapidly forming stars.  The background image in both frames is the
GRS \thirteenco image integrated over the range $40\kms < v_{lsr} < 75 \kms$,
masked to include only pixels with $S>0.5$ K. \todo{Unclosed contours should
probably be ignored or removed.}}
{fig:dgmf_contours}{1.0}{6.5in}

The multi-density DGMF presented here can be compared to models of `global'
collapse.  They are effectively a gas density cumulative distribution function.
However, to understand the systematic effects of line-of-sight stacking of
different velocity components (and corresponding radiative transfer issues),
similar analysis should be performed on hydrodynamic simulations.

\subsubsection{Dense Gas Fraction assumptions \& caveats}

This analysis relies on the \thirteenco being optically thing and thermally
excited, both of which are generally good assumptions for the majority of the
mass.  The molecular cloud probably includes no more than $\sim20\%$ of its mass
in CO-dark gas \citep{Pineda2013a,Langer2013a,Smith2014b}, which adds little to
the overall uncertainty.

In Section \ref{sec:densmaps}, we discussed the various caveats and issues
related to \formaldehyde density fitting.  To account for the full range of
errors in that analysis, we have plotted the DGMF calculated using the minimum
and maximum values of the \formaldehyde-derived density consistent with the
data at the $1-\sigma$ level in Figure \ref{fig:dgmf} and
\ref{fig:dgmf_regions}.

We have assigned \emph{all} of the mass associated with a given PPV voxel with
a single, fixed density in this analysis.  There is certainly some mass at a
lower density in each PPV pixel.  This additional mass biases the measured DGMF
higher than it should be, but probably only by a small amount at each
threshold.  This systematic bias could be characterized from simulations
projected into PPV space.

\subsection{Implications for the future evolution of W51}
The low DGMF associated with the W51 B / 66 \kms cloud indicates that it has a
low star formation potential despite containing significant mass or, at least,
being quite bright in CO and dust emission.  The presence of a supernova
remnant and old, diffuse HII regions indicates that the cloud did previously
(and recently) form stars, but is now being destroyed.  This cloud is therefore
a good region to examine the effects of different feedback mechanisms in
parallel.  It may also be a good location to examine how star formation
progresses, if it does at all, in the presence of severe feedback and at the
tail end of the overall cloud-to-star process.

By contrast, the W51 Main region has a dense gas fraction $\sim1$, or at least
$\sim50\%$ out to nearly 5 pc.  It has presumably only formed a small fraction
of its total potential.

\subsection{Implications for extragalactic observations of \formaldehyde}
Although W51 is one of the most massive and active GMCs in the galaxy,
containing 7\% of the present-day massive star formation galaxy-wide
\citep{Urquhart2014a}, its star-forming gas mass is predominantly at a moderate
density, $n\sim5\ee{4}$ \percc; there is very little gas above $10^6$ \percc even
in W51 Main.  There were \emph{no detections} of \formaldehyde emission on the
$\sim1.25$ pc (50\arcsec) scales observed.

By contrast, in extragalactic observations of starburst galaxies, there have
been detections of emission on $\sim100$ pc scales.  \citet{Mangum2013a} report
detections of \formaldehyde \oneone emission in NGC 3079, IC 860, IR
15107+0724, and Arp 220 on $\sim10$ kpc scales.  The implied local column
densities from their analysis are modest, but the densities are extreme: their
observations imply that the local-scale \emph{chemical} conditions are
comparable to W51, but the densities are different.

The only location in which densities $n\gtrsim5\ee{5}$ \percc (comparable to
$n_{\hh}(\textrm{Arp } 220)$, etc.) are observed in W51 are in the central W51
Main region.  We don't observe emission because of the extremely bright
continuum background source.  It is therefore not possible to explain a
\formaldehyde-emission galaxy by constructing it from collections of UCHII
regions; such a galaxy would be continuum-bright and show only \formaldehyde
absorption.  Instead, they must be assembled from huge quantities of
high-density, non-star-forming gas.  This result is in contradiction to the
idea that Giant \hii Regions are the `building blocks' of starburst galaxies
\citep[e.g.][]{Miura2014a}.  


\section{Conclusion}
We have presented maps of the \formaldehyde \oneone and \twotwo and H77$\alpha$
and H110$\alpha$ lines covering the W51 star forming complex.  The continuum
data were compared with previous lower-resolution observations of the region,
resulting yielding a consistent calibration.

The recombination lines have been used in conjunction with the continuum to
estimate electron temperatures, with a consistent mean $T_e^*\sim7500$ K.

The \formaldehyde \oneone/\twotwo line ratio was used to measure gas volume
densities.  While systematic uncertainties in the modeling remain, the
\formaldehyde nonetheless yields a consistent picture of the W51 star forming
region in which the central W51 Main proto-cluster is filled with gas of a mean
density at least an order of magnitude greater than the rest of the cloud.

The \formaldehyde lines and their ratios have also been used to constrain the
geometry of the W51 GMC and the associated \hii regions.  The Galactic face-on
view of W51 is presented in more detail than has previously been possible.

The data are made available in FITS cubes and images hosted at the CfA
dataverse \url{doi:10.7910/DVN/26818},
\url{http://thedata.harvard.edu/dvn/dv/W51_H2CO}.
The entire reduction and analysis process and all associated code and scripts
are made available via a git
repository hosted on github:
\url{https://github.com/keflavich/w51_singledish_h2co_maps}, \todo{with a
snapshot of the publication version available as a tarball from figshare
\url{TBD}}.



 
% \FigureTwo{figures/MCMC_DensColplot_67_64.png}{figures/spec67_64_bestfit_MCMC.png}
% {Plots demonstrating upper limit fits.  The left plot shows the allowed
% parameter space from MCMC sampling of the data given the RADEX model.  The
% right plot shows the `best-fit' model to the optical depth spectra, which is
% clearly unconstrained by the relatively insensitive \twotwo\ spectrum.  The
% sensitivity in the \oneone line is better in large part because of brighter 6
% cm background across the whole W51 region.  Despite the lack of constraint on the
% volume density, there is a reasonably strong constraint on the column density.}
% {fig:w51MCMCcompare}{1}
% 
% The molecular gas is concentrated near, but not exactly on, the bright cm
% peaks.  W51 IRS2 has a massive clump of gas at 65 \kms, and W51 e2 has a
% similar clump.  However, e2 also seems to have a very dense ($n>10^5 \percc$)
% infalling clump.  The spectra, along with multicomponent fits, are shown in
% Figure \ref{fig:w51hiispectra}.
% 
% \FigureTwo{figures/W51_bestfit_spec53_49_IRS2.png}{figures/W51_bestfit_spec53_49_W51e2.png}
% {Plots of the optical depth spectra centered on W51 IRS2 (left) and W51 e2, an
% ultracompact HII region (right).  IRS2 shows high-density gas with a slight
% hint of infall, but otherwise a somewhat vanilla spectrum.  W51e2 has a large,
% high-density red shoulder, indicating high-density gas at the most red velocity
% in the system.  Because this is foreground gas, that high-density gas
% \emph{must} be moving towards the \uchii region.}
% {fig:w51hiispectra}{1}

\textbf{Acknowledgements}:
We thank Xiaohui Sun for providing the Urumqi 6 cm Stokes I image prior to its
availability on the survey website.

\textbf{Code Packages Used}:

\begin{itemize}
    \item The GBT KFPA Pipeline \url{https://safe.nrao.edu/wiki/bin/view/Kbandfpa/ObserverGuide}
    \item aoIDL \url{http://www.naic.edu/~phil/download/aoIdl.tar.gz}
    \item astropy \url{www.astropy.org}
    \item astroquery \url{astroquery.readthedocs.org}
    \item sdpy \url{https://github.com/keflavich/sdpy}
    \item \texttt{FITS\_tools} \url{https://github.com/keflavich/FITS_tools}
    \item aplpy \url{http://aplpy.github.io}
    \item image-registration \url{http://image-registration.rtfd.org}
\end{itemize}

%\appendix
%\section{A complete record of the data}
%In the interest of reproducibility, we include a complete code suite detailing
%all of the steps from raw data to final reduced product.  If one acquires all
%of the raw data from both Arecibo and GBT, the data and derivative products
%from this publication can all be reproduced in their entirety.
%
%The Arecibo data was processed using code based on the aoIDL package by Phil
%Perillat.  The \texttt{accum\_map} routine performs the standard on-off reduction
%steps and stores the resulting time-series of spectra in a FITS file.  The
%\texttt{make\_off} routine creates an `off' spectrum by averaging selected
%(ideally emission-free) scans, smoothing them, and interpolating across
%regions of the spectrum with lines or electronic artifacts.  The data were then
%gridded into map cubes using the \texttt{makecube} routines in \texttt{sdpy}.
%
%The GBT data was processed using exclusively custom-made scripts stored in
%\texttt{sdpy}.  The KFPA pipeline
%(\url{https://safe.nrao.edu/wiki/bin/view/Kbandfpa/ObserverGuide}) is a useful
%alternative, but at the time these data were taken, it was not quite mature
%enough for use with our Ku-band data.  The custom-made scripts also provided
%needed flexibility for comparing the GBT and Arecibo data.




/Users/adam/work/h2co/maps/paper/solobib.tex
\end{document}

