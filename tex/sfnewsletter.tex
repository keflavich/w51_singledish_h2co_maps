%%%%%%%%%%%%%%%%%%%%%%%%%%%%%%%%%%%%%%%%%%%%%%%%%%%%%%%%%%%%%%%%%%%%%%%%%%%%%
%%%                                                                       %%%
%%%            LaTeX MACRO FOR THE STAR FORMATION NEWSLETTER              %%%
%%%                                                                       %%%
%%%    Please use for abstracts of papers which have been ACCEPTED in     %%%
%%%    REFEREED JOURNALS (do not send abstracts of reviews for books      %%%
%%%    or conference notes).  Merely fill in the brackets below and       %%%
%%%    mail to reipurth@ifa.hawaii.edu.  If you have problems, let me     %%%
%%%    know in an accompanying note and I will fix them.                  %%%
%%%                                                                       %%%
%%%%%%%%%%%%%%%%%%%%%%%%%%%%%%%%%%%%%%%%%%%%%%%%%%%%%%%%%%%%%%%%%%%%%%%%%%%%%

\documentclass[]{article}
\textwidth 18cm
\textheight 23cm
\oddsidemargin -1cm
\topmargin 0cm
\parskip 0.15cm
\parindent 0pt
\small

% needed for gtrsim:
\usepackage{amssymb}
% Needed for the DOI url
\usepackage{url}
% needed for other macros
\usepackage{xspace}
% Macros for author institutions
\newcommand{\eso}{$^{1}$}
\newcommand{\casa}{$^{2}$}
\newcommand{\cfa}{$^{3}$}
\newcommand{\edmonton}{$^{4}$}
\newcommand{\yale}{$^{5}$}
\newcommand{\puertorico}{$^{6}$}

\begin{document}

{\large\bf{The dense gas mass fraction in the W51 cloud and its protoclusters}}

{\bf{
Adam Ginsburg{\eso},
John Bally{\casa},
Cara Battersby{\cfa},
Allison Youngblood{\casa},
Jeremy Darling{\casa},
Erik Rosolowsky{\edmonton},
Héctor Arce{\yale},
Mayra E. Lebrón Santos{\puertorico}
}}

{\eso}{\it{European Southern Observatory, Karl-Schwarzschild-Strasse 2, D-85748 Garching bei München, Germany }} \\ 
{\casa}{\it{CASA, University of Colorado, 389-UCB, Boulder, CO 80309}} \\ 
{\cfa}{\it{Harvard-Smithsonian Center for Astrophysics, 60 Garden Street, Cambridge, MA 02138, USA}} \\ 
{\edmonton}{\it{University of Alberta, Department of Physics, 4-181 CCIS, Edmonton AB T6G 2E1 Canada}} \\ 
{\yale}{\it{Department of Astronomy, Yale University, P.O. Box 208101, New Haven, CT 06520-8101 USA}} \\ 
{\puertorico}{\it{Department of Physical Sciences, University of Puerto Rico, P.O. Box 23323, San Juan, PR 00931}}

%% Here you may write the e-mail address of one or more of the authors
%% who will act as contact person for preprint requests etc, for example:

{E-mail contact: adam.g.ginsburg@gmail.com}


  %% IF YOU USE ANY PERSONAL LATEX COMMANDS IN YOUR ABSTRACT,
  %% PLEASE INCLUDE THEIR DEFINITIONS HERE!
  %% AND PLEASE INCLUDE ONLY THOSE YOU NEED FOR THE ABSTRACT.

\newcommand{\percc}{\ensuremath{\textrm{cm}^{-3}}\xspace}
\newcommand{\formaldehyde}{\ensuremath{\textrm{H}_2\textrm{CO}}\xspace}
\newcommand{\hh}{\ensuremath{\textrm{H}_{2}}\xspace}			%  H2
\newcommand{\hii}{H~{\sc ii}\xspace}
\def\ee#1{\ensuremath{\times10^{#1}}}
\newcommand\arcsec{\mbox{$^{\prime\prime}$}\xspace} 

%% Within the following brackets you place your text:

{

\textit{Context:} The density structure of molecular clouds determines how they will evolve.
\textit{Aims:} To map the velocity-resolved density structure of the most vigorously
star-forming molecular cloud in the Galactic disk, the W51 Giant Molecular
Cloud.\\
\textit{Methods:}
We present new 2 cm and 6 cm maps of \formaldehyde, radio recombination lines,
and the radio continuum in the W51 star forming complex acquired with Arecibo
and the Green Bank Telescope at $\sim50\arcsec$ resolution.
We use \formaldehyde absorption to determine the relative line-of-sight
positions of molecular and ionized gas.  We measure gas densities using the
\formaldehyde densitometer, including continuous measurements of the dense gas
mass
fraction (DGMF) over the range $10^4$ $\percc< n(\hh) < 10^6$ \percc - this is
the first time a dense gas mass fraction has been measured over a range
of densities with a single data set.\\
\textit{Results:} The DGMF in W51 A is high, $f\gtrsim70\%$ above $n>10^4$
\percc, while it is low, $f<20\%$, in W51 B.
We did not detect \emph{any} \formaldehyde emission throughout the W51 GMC; all
gas dense enough to emit under normal conditions is in front of bright
continuum sources and therefore is seen in absorption instead.  
The data set has been made public at \protect\url{http://dx.doi.org/10.7910/DVN/26818}. \\
\textit{Conclusions:} (1) The dense gas fraction in the W51 A and B clouds shows that W51 A will
continue to form stars vigorously, while star formation has mostly ended in W51
B.  The lack of dense, star-forming gas around W51 C indicates that
collect-and-collapse is not acting or is inefficient in W51.
(2) Ongoing high-mass star formation is correlated with $n\gtrsim1\ee{5}$
\percc gas.  Gas with $n>10^4$ \percc is weakly correlated with low and
moderate mass star formation, but does not strongly correlate with high-mass
star formation.
(3) The nondetection of \formaldehyde emission implies that the  emission
detected in other galaxies, e.g. Arp 220, comes from high-density gas that is
not directly affiliated with already-formed massive stars.  Either the
non-star-forming ISM of these galaxies is very dense, implying
the star formation density threshold is higher, or \hii regions have
their emission suppressed.

}


% Here you write which journal accepted your paper, for example:

{ Accepted by A\&A }

%% If preprints are available on the WWW you can give the web
%% direction here.
%% (will be posted on arXiv soon)


\end{document}
